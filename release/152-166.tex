%Чернобаев

\section*{Раздел 29.5: Реализация на Java}
\begin{tcolorbox}
\begin{minted}[obeytabs=true,tabsize=3,style = vs]{java}
public class MyBubbleSort {
	// функция сортировки пузырьком
	public static void bubble_srt(int array[]) {
		int n = array.length; // количество элементов
		int k;
		for (int m = n; m >= 0; m--) { // обратная итерация по массиву
			// прямая итерация, без последнего элемента
			for (int i = 0; i < n - 1; i++) {
				// перебор по индексу k + i
				k = i + 1;
				// если соседние элементы не по порядку
				if (array[i] > array[k]) {
					// меняем местами
					swapNumbers(i, k, array);
				}
			}
			// выводим
			printNumbers(array);
		}
	}
	// функция перестановки двух чисел в массиве
	private static void swapNumbers(int i, int j, int[] array) {
		int temp; // буфер для временного значения
		temp = array[i];
		array[i] = array[j];
		array[j] = temp;
	}
	// функция вывода массива
	private static void printNumbers(int[] input) {
		// итерация по всему массиву
		for (int i = 0; i < input.length; i++) {
			System.out.print(input[i] + ", "); // вывод по элементно
		}
		System.out.println("\n");
	}
	public static void main(String[] args) {
		// тестовые данные	
		int[] input = { 4, 2, 9, 6, 23, 12, 34, 0, 1 };
		// заход в главную функцию
		bubble_srt(input);
	}
}
\end{minted}
\end{tcolorbox}
\section*{Раздел 29.6: Реализация на Javascript}
\begin{tcolorbox}
\begin{minted}[obeytabs=true,tabsize=3,style = vs]{java}
function bubbleSort(a) {
	var swapped; // флаг произошедшей рокировки
	do {
		swapped = false; // обнуление флага
		for (var i=0; i < a.length-1; i++) { // проход по всему массиву
			if (a[i] > a[i+1]) { // если соседние элементы не по порядку
				var temp = a[i]; // записываем элемент во временную переменную
				a[i] = a[i+1]; // меняем значения переменных местами
				a[i+1] = temp;
				swapped = true; // рокировка произошла
			}
		}
	} while (swapped); // пока все элементы не по порядку
}
var a = [3, 203, 34, 746, 200, 984, 198, 764, 9]; // тестовый набор
bubbleSort(a);
// результат
console.log(a); //logs [ 3, 9, 34, 198, 200, 203, 746, 764, 984 ]
\end{minted}
\end{tcolorbox}
\chapter*{Глава 30: Сортировка слиянием}
\section*{Раздел 30.1: Основы сортировки слиянием}
Сортировка слиянием - это алгоритм типа «разделяй и властвуй». Он делит входной список длиной n пополам до тех пор, пока не будет n списков размером 1. Затем пары списков объединяются вместе с меньшим первым элементом из пары списков, добавленных на каждом шагу. Посредством последовательного слияния и сравнения первых элементов создается отсортированный список.

\textbf{Пример:}

\vspace{\baselineskip}

\includeimage{0.7}{images/154_1.pdf}\\
\textbf{Временная сложность:}
\colorbox{gray!6!white}{ \parbox{10em}{$T(n)= 2T(n/2) + \Theta(n)$}}

\vspace{\baselineskip}

Вышеуказанная рекуррентность может быть решена с использованием метода дерева рекурсивных вызовов или метода рекуррентных соотношений. Это относится к случаю II метода рекуррентных соотношений, и решение получается за $\theta(nLogn)$. Временная сложность сортировки слиянием составляет $\theta(nLogn)$ во всех 3 случаях (наихудший, средний и лучший), поскольку сортировка слиянием всегда делит массив на две половины и требует линейного времени для объединения двух половин.

\textbf{Вспомогательное пространство:} \colorbox{gray!6!white}{ \parbox{2em}{O(n)}}

\vspace{\baselineskip}

\textbf{Алгоритмическая парадигма:} Разделяй и властвуй

\vspace{\baselineskip}

\textbf{Сортировка на месте:} Нетипичная реализация

\vspace{\baselineskip}

\textbf{Стабильно:} Да

\vspace{\baselineskip}

\section*{Раздел 30.2: Реализация сортировки слиянием на Go}
\begin{tcolorbox}
\begin{minted}[obeytabs=true,tabsize=3,style = vs]{go}
package main
import "fmt"
func mergeSort(a []int) []int {
	// базовый случай - один элемент в массиве - конечный результат
	if len(a) < 2 {
		return a
	}
	// медианный индекс
	m := (len(a)) / 2
	// левая часть массива
	f := mergeSort(a[:m])
	// правая часть
	s := mergeSort(a[m:])
	// возвращаем через рекурсию более мелкие части массива до 1 элемента
	return merge(f, s)
}
func merge(f []int, s []int) []int {
	var i, j int
	// сумма сравниваемых частей массива
	size := len(f) + len(s)
	// выделяем память под результирующий массив
	a := make([]int, size, size)
	// итерация по всему массиву
	for z := 0; z < size; z++ {
		lenF := len(f) // длина первой части
		lenS := len(s) // длина второй части
		if i > lenF-1 && j <= lenS-1 { // элементов второго массива больше
			a[z] = s[j] // присваиваем недостающий элемент
			j++ 
		} else if j > lenS-1 && i <= lenF-1 { // первый массив больше
			a[z] = f[i] // присваиваем недостающий элемент
			i++ 
		} else if f[i] < s[j] { // элемент второго массива больше
			a[z] = f[i] // присваиваем элемент первого
			i++ // элемент первого массива больше
		} else { 
			a[z] = s[j] // присваиваем элемент второго
			j++
		}
	}
	return a // возвращаем результатирующий массив
}
func main() {
	// тестовые 
	a := []int{75, 12, 34, 45, 0, 123, 32, 56, 32, 99, 123, 11, 86, 33}
	// вывод исходных данных
	fmt.Println(a)
	// вывод результата
	fmt.Println(mergeSort(a))
}
\end{minted}
\end{tcolorbox}
\section*{Раздел 30.3: Реализация сортировки слиянием на C $\&$ C$\#$}
\textbf{Сортировка слиянием на C}

\begin{tcolorbox}
\begin{minted}[obeytabs=true,tabsize=3,style = vs]{c}
int merge(int arr[],int l,int m,int h) {
	// Два временных массива для объединения
	int arr1[10],arr2[10];	 
	int n1,n2,i,j,k;
	n1=m-l+1; // длина левой части
	n2=h-m; // длина правой части
	for(i=0; i<n1; i++) // записываем в буфер левую часть
		arr1[i]=arr[l+i];
	for(j=0; j<n2; j++)
		arr2[j]=arr[m+j+1]; // записываем в буфер правую часть
	arr1[i]=9999; // отметить конец каждого временного массива
	arr2[j]=9999;
	i=0;pulseaudio -k && sudo alsa force-reload
	j=0;
	for(k=l; k<=h; k++) { // объединение двух отсортированных массивов
		if(arr1[i]<=arr2[j]) // меньший элемент присваиваем результату
			arr[k]=arr1[i++];
		else
			arr[k]=arr2[j++];
	}
	return 0;
}
int merge_sort(int arr[],int low,int high) {
	int mid;
	if(low<high) {
		mid=(low+high)/2; // медианный индекс
		// Разделение 
		merge_sort(arr,low,mid); // на левую
		merge_sort(arr,mid+1,high); // на правую
		merge(arr,low,mid,high); // слияние 
	}
	return 0;
}
\end{minted}
\end{tcolorbox}

\vspace{\baselineskip}

Сортировка Слиянием на C$\#$ \\
\begin{tcolorbox}
\begin{minted}[obeytabs=true,tabsize=3,style = vs]{csharp}
public class MergeSort {
	static void Merge(int[] input, int l, int m, int r) {
		int i, j;
		var n1 = m - l + 1; // длина левой части
		var n2 = r - m; // длина правой части
		var left = new int[n1]; // выделяем память под два массива
		var right = new int[n2];
		for (i = 0; i < n1; i++) {
			left[i] = input[l + i]; // заполняем левыми элементами
		}
		for (j = 0; j < n2; j++) {
			right[j] = input[m + j + 1]; // заполняем правыми элементами
		}
		i = 0;
		j = 0;
		var k = l;
		while (i < n1 && j < n2) {
			if (left[i] <= right[j]) { // элемент левого массива меньше
				input[k] = left[i]; // заполняем им
				i++;
			}
			else {
				input[k] = right[j]; // иначе элемент правого массива
				j++;
			}
			k++;
		}
		 // один из следующих циклов будет пропущен
		while (i < n1) { // остается добавить либо левую часть
			input[k] = left[i];
			i++;
			k++;
		}
		while (j < n2) { // либо правую
			input[k] = right[j];
			j++;
			k++;
		}
	}
	static void SortMerge(int[] input, int l, int r) {
		if (l < r) {
			int m = l + (r - l) / 2; // медианный индекс
			SortMerge(input, l, m); // рекурсивно разбиваем правую часть
			SortMerge(input, m + 1, r); // рекурсивно разбиваем левую часть
			Merge(input, l, m, r); // сливаем части
		}
	}
	public static int[] Main(int[] input) {
		SortMerge(input, 0, input.Length - 1);	// начало сортировки
		return input;
	}
}
\end{minted}
\end{tcolorbox}
\section*{Раздел 30.4: Реализация сортировки слиянием на Java}
Ниже приводится реализация на Java с использованием обобщенного программирования. Это тот же алгоритм, который представлен выше.\\
\begin{tcolorbox}
\begin{minted}[obeytabs=true,tabsize=3,style = vs]{java}

public interface InPlaceSort<T extends Comparable<T>> {
	void sort(final T[] elements); // шаблон элементов массива
}
public class MergeSort < T extends Comparable < T >> implements InPlaceSort < T > {

	@Override
	public void sort(T[] elements) {
		 // объявляем сравниваемый массив
		T[] arr = (T[]) new Comparable[elements.length];
		// начинаем сортировку
		sort(elements, arr, 0, elements.length - 1);
	}
	// Мы проверяем обе наши стороны и затем объединяем их
	private void sort(T[] elements, T[] arr, int low, int high) {
		if (low >= high) return; // в случае пустого массива
		int mid = low + (high - low) / 2; // медианный элемент
		sort(elements, arr, low, mid); // рекурсия левой части
		sort(elements, arr, mid + 1, high); // рекурсия правой части
		merge(elements, arr, low, high, mid); // слияние
	}
	private void merge(T[] a, T[] b, int low, int high, int mid) {
		int i = low; // начало левого массива
		int j = mid + 1; // начало правого массива
		// выбираем самый меньший элемент из двух, помещаем в массив
		for (int k = low; k <= high; k++) {
			if (i <= mid && j <= high) {
				if (a[i].compareTo(a[j]) >= 0) { // вычитание правого из левого
					b[k] = a[j++];
				} else { // иначе, левый элемент больше
					b[k] = a[i++];
				}
			} else if (j > high && i <= mid) { // один из массивов закончился 
				b[k] = a[i++];
			} else if (i > mid && j <= high) {
				b[k] = a[j++];
			}
		}
		for (int n = low; n <= high; n++) {
		a[n] = b[n]; // запись результата в первый массив 
		}
	}
}
\end{minted}
\end{tcolorbox}
\section*{Раздел 30.5: Реализация сортировки слиянием на Python}
\begin{tcolorbox}
\begin{minted}[obeytabs=true,tabsize=3,style = emacs]{python}
def merge(X, Y):
	" слияние двух отсортированных массивов "
	p1 = p2 = 0
	out = [] # конечный массив
	while p1 < len(X) and p2 < len(Y):
		if X[p1] < Y[p2]:
			out.append(X[p1]) # записываем меньшие элементы из двух исходных
			p1 += 1 # инкремент индекса
		else:
			out.append(Y[p2])
			p2 += 1 # инкремент второго индекса
	out += X[p1:] + Y[p2:] # запись остатка
	return out
def mergeSort(A):
	if len(A) <= 1: # базовые случаи
		return A
	if len(A) == 2:
		return sorted(A)
	mid = len(A) / 2 # медианное значение
	return merge(mergeSort(A[:mid]), mergeSort(A[mid:])) # рекурсия
if __name__ == "__main__":
	 # генерируем 20 элементов тестовых данных
	A = [randint(1, 100) for i in xrange(20)] print mergeSort(A)
\end{minted}
\end{tcolorbox}
\section*{Раздел 30.6: Реализация на Java метода восходящего слияния} 
\begin{tcolorbox}
\begin{minted}[obeytabs=true,tabsize=3,style = vs]{java}
public class MergeSortBU {
	// тестовые данные
	private static Integer[] array = { 
		4, 3, 1, 8, 9, 15, 20, 2, 5, 6, 30, 70,60,80,0,9,67,54,51,52,24,54,7		 };
	public MergeSortBU() {
	}
	private static void merge(Comparable[] arrayToSort, Comparable[] aux, int lo,int mid, int hi) {
	 // делаем копию
	for (int index = 0; index < arrayToSort.length; index++) {
		aux[index] = arrayToSort[index];
	}
	int i = lo; // нижний индекс
	int j = mid + 1; // медиана
	for (int k = lo; k <= hi; k++) {
		if (i > mid) // когда дошли до середины идним из массивов
			arrayToSort[k] = aux[j++];
		else if (j > hi)
			arrayToSort[k] = aux[i++];
		 // выполняется сначала нижняя часть составных условий
		else if (isLess(aux[i], aux[j])) {
			// присваиваем меньший элемент, увеличиваем счетчик

			arrayToSort[k] = aux[i++];
		} else {
			arrayToSort[k] = aux[j++];
		}
	}
}
public static void sort(Comparable[] arrayToSort, Comparable[] aux, int lo, int hi) {
	int N = arrayToSort.length; // количество элементов
	 // цикл с шагом в два раза большим предыдущего
	for (int sz = 1; sz < N; sz = sz + sz) {
		 // шаг больше на величину индекса, прошлого элемента
		for (int low = 0; low < N; low = low + sz + sz) {
			System.out.println("Size:"+ sz); // вывод размера
			 // слияние непересекающихся частей
			merge(arrayToSort, aux, low, low + sz -1, Math.min(low + sz + sz - 1, N - 1));
			print(arrayToSort); // вывод результата
		}
	}
}
public static boolean isLess(Comparable a, Comparable b) {
	return a.compareTo(b) <= 0; // истина, если первый элемент меньше
}
private static void print(Comparable[] array) {
	StringBuffer buffer = new StringBuffer(); // объявляем для копии
		for (Comparable value : array) { // заполняем с пробелами
			buffer.append(value);
			buffer.append(' ');
		}
		System.out.println(buffer); // выводим копию
	}
	public static void main(String[] args) {
		 // объявляем обрабатываемый массив
		Comparable[] aux = new Comparable[array.length];
		print(array);
		MergeSortBU.sort(array, aux, 0, array.length - 1);
	}
}
\end{minted}
\end{tcolorbox}
\newpage
\chapter*{Глава 31: Сортировка вставками } 

\section*{Раздел 31.1: Реализация на Haskell}
\begin{tcolorbox}
\begin{minted}[obeytabs=true,tabsize=3,style = emacs]{haskell}
	-- делаем копию массива
insertSort :: Ord a => [a] -> [a]
insertSort [] = []
-- перебор по исходному с присваиванием
insertSort (x:xs) = insert x (insertSort xs)
-- начало условия включения
insert :: Ord a => a-> [a] -> [a]
-- включить остаток массива
insert n [] = [n]
-- который меньше чем текущий
insert n (x:xs) | n <= x = (n:x:xs)
								-- иначе обратная часть
								| otherwise = x:insert n xs
\end{minted}
\end{tcolorbox}


\newpage
\chapter*{Глава 32: Блочная сортировка} 

\section*{Раздел 32.1: Реализация на C$\#$}
\begin{tcolorbox}
\begin{minted}[obeytabs=true,tabsize=3,style = vs]{csharp}
public class BucketSort {
	public static void SortBucket(ref int[] input) {
		int minValue = input[0]; // начальные пограничные значения
		int maxValue = input[0];
		int k = 0;
		 // обратный цикл по массиву
		for (int i = input.Length - 1; i >= 1; i--) {
			 // поиск максимального и минимального значения
			if (input[i] > maxValue) 
				maxValue = input[i];
			if (input[i] < minValue) 
				minValue = input[i];
		}
		 // объявление двойного массива индексов значений
		List<int>[] bucket = new List<int>[maxValue - minValue + 1];
		for (int i = bucket.Length - 1; i >= 0; i--) {
			bucket[i] = new List<int>(); // объявление для индексов чисел
		}
		foreach (int i in input) {
			bucket[i - minValue].Add(i); // добавляем индексы ключа-значения
		}
		foreach (List<int> b in bucket) {
			if (b.Count > 0) {
				 // если элемент присутствует в подмассиве индексов
				foreach (int t in b) {
					input[k] = b[t];	// заполняем значениями в выходной массив
					k++;
				}
			}
		}
	}
	public static int[] Main(int[] input) {
		SortBucket(ref input);	// в функцию указатель на исходный массив
		return input; // значение изменилось на результат
	}
}
\end{minted}
\end{tcolorbox}

\newpage
\chapter*{Глава 33: Быстрая сортировка}

\section*{Раздел 33.1: Основы быстрой сортировки}
%\vspace{-1cm}
\href{https://en.wikipedia.org/wiki/Quicksort}{\underline{{Быстрая сортировка}}} - это алгоритм сортировки, который выбирает элемент («опорный элемент») и переупорядочивает массив, образуя два разбиения, так что все элементы, меньше опорного, идут перед ним, а все элементы с большим значением идут после него. Затем алгоритм рекурсивно применяется к разбиениям до тех пор, пока список не будет отсортирован.

\vspace{\baselineskip}

\textbf{1. Механизм схемы разбиения Ломуто:}

\vspace{\baselineskip}

Эта схема выбирает опорный элемент, который обычно является последним элементом в массиве. Алгоритм сохраняет индекс, чтобы поместить опорный элемент в переменную i, и каждый раз, когда он находит элемент, меньший или равный опорному элементу, этот индекс увеличивается, и этот элемент будет помещен перед опорным.
\begin{tcolorbox}[breakable,enhanced,before upper={\parindent10pt}]
\noindent partition(A, low, high) is

pivot := A[high] // опорный элемент

i := low // индекс начала

for j := low to high - 1 do

\indent if A[j] <= pivot then // если текущий элемент меньше опорного

\indent\indent swap A[i] with A[j] // меняем их местами

\indent\indent i := i + 1 // смещаем индекс начала

swap A[i] with A[high]	// меняем последний элемент с обрабатываемым

return i // возвращаем индекс обрабатываемого элемента

\end{tcolorbox}

\vspace{\baselineskip}

Механизм быстрой сортировки:
\begin{tcolorbox}[breakable,enhanced,before upper={\parindent10pt}]
\noindent quicksort(A, low, high) is

if low < high then // обработка в заданном диапазоне

\indent p := partition(A, low, high) 

\indent quicksort(A, low, p - 1) // рекурсивный обход с левой частью

\indent quicksort(A, p + 1, high) // с правой частью

\end{tcolorbox}

\vspace{\baselineskip}

\textbf{Пример быстрой сортировки:}

\includeimage{0.8}{images/164_1.pdf}

\vspace{\baselineskip}

\textbf{2. Схема разбиения Хоара:}

\vspace{\baselineskip}

Она использует два индекса, которые начинаются на концах разбиваемого массива, а затем движутся навстречу друг другу, пока не обнаружат инверсию: пара элементов, один из которых больше или равен опорному, другой меньше или равен опорному, и эта пара неправильно упорядочена относительно друг друга. Затем инвертированные элементы меняются местами. Когда индексы встречаются, алгоритм останавливается и возвращает конечный индекс. Схема Хоара более эффективна, чем схема разбиения Ломуто, потому что она делает в среднем в три раза меньше перестановок и создает эффективные разбиения, даже если все значения равны.

\begin{tcolorbox}[breakable,enhanced,before upper={\parindent10pt}]
\noindent quicksort(A, lo, hi) is\\
if lo < hi then// обработка в заданном диапазоне\\
\indent p := partition(A, lo, hi)\\
\indent quicksort(A, lo, p) // рекурсивный обход с левой частью\\
\indent quicksort(A, p + 1, hi) // с правой частью\\
\end{tcolorbox}

\vspace{\baselineskip}

Разбиение :

\vspace{\baselineskip}

\begin{tcolorbox}[breakable,enhanced,before upper={\parindent10pt}]
\noindent partition(A, lo, hi) is\\
pivot := A[lo] // опорный первый элемент\\
i := lo - 1

j := hi + 1

loop forever do:

\indent i := i + 1 // прямой перебор, сдвигаем левую границу вправо\\
\indent while A[i] < pivot do: // пока меньше опорного\\
\indent\indent j := j - 1 // сдвигаем правую границу влево\\
\indent while A[j] > pivot do: // пока правый граничный больше опорного \\
\indent\indent if i >= j then // если границы сомкнулись\\
\indent\indent\indent return j // возвращаем граничный индекс элемента\\
\indent swap A[i] with A[j] // меняем местами\\
\end{tcolorbox}

\section*{33.2: Реализация быстрой сортировки на Python}
\begin{tcolorbox}
\begin{minted}[obeytabs=true,tabsize=3,style = emacs]{python}
def quicksort(arr):
	if len(arr) <= 1: # базовый случай
		return arr
	pivot = arr[len(arr) / 2] # опорный элемент посередине
	left = [x for x in arr if x < pivot] # все элементы, меньшие опорного
	middle = [x for x in arr if x == pivot] # все равные опорному
	right = [x for x in arr if x > pivot]	# все большие опорного
	# рекурсивно обрабатываются две части
	return quicksort(left) + middle + quicksort(right)

print quicksort([3,6,8,10,1,2,1])
\end{minted}
\end{tcolorbox}
\textbf{Печатает "[1, 1, 2, 3, 6, 8, 10]"}
\section*{Раздел 33.3: Реализация разбиения Ломуто на Java}
\begin{tcolorbox}
\begin{minted}[obeytabs=true,tabsize=3,style = vs]{java}
public static void main(String[] args) {
		Scanner sc = new Scanner(System.in);	// вводим массив
		int n = sc.nextInt();
		int[] ar = new int[n]; // объявляем обрабатываемый массив
		for(int i=0; i<n; i++)
			ar[i] = sc.nextInt(); // заполняем
		quickSort(ar, 0, ar.length-1);
	}
	public static void quickSort(int[] ar, int low, int high) {
		if(low<high) { // если границы не сомкнулись
			int p = partition(ar, low, high); // разбиваем с опорным элементом
			quickSort(ar, 0 , p-1); // обработка левой части
			quickSort(ar, p+1, high); // правой части
		}
	}
	public static int partition(int[] ar, int l, int r) {
		int pivot = ar[r]; // опорный последний элемент
		int i = l;
		for(int j=l; j<r; j++) {	// начиная слевой границы
			if(ar[j] <= pivot) { // элемент меньше опорного
				int t = ar[j]; // меняем местами текущий с опорным
				ar[j] = ar[i];
				ar[i] = t;
				i++; // шаг для левой границы
			}
		}
		int t = ar[i]; // смена местами
		ar[i] = ar[r];
		ar[r] = t;
	return i;
	}
}
\end{minted}
\end{tcolorbox}














