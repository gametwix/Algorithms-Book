%Арапов
\vspace{\baselineskip}
\begin{tcolorbox}
\begin{minted}{C}
	 while(!rootFound){
		for(i=0; i<n; i++){		
			Nx[i]=b[i];
			
			for(j=0; j<n; j++){
				if(i!=j) Nx[i]= Nx[i]-a[i][j]*Nx[j];
			}
			x[i]= Nx[i]/ a[i][i];
		}
		
		rootFound=1;				
		for(i=0; i<n; i++){
			if(!((Nx[i]-x[i])/x[i]>-0.000001&&(Nx[i]-x[i])/x[i]<0.000001)){
				rootFound=0;
				break;
			}
		}
		
		for(i=0; i<n; i++){		
			x[i]=Nx[i];
		}
	}
	
	return ;
}


void print(int n, double x[n]){
	int i;
	for(i=0; i<n; i++){
		rintf("%lf, ", x[i]);
	}
	printf("\n\n");
	
	return ;
}

int main(){
	
	int n=3;			
	double x[n];		
	double b[n],		
		[n][n];		 
		
	[0][0]=8; a[0][1]=2; a[0][2]=-2; b[0]=8;		//8x_1+2x_2-2x_3+8=0
	a[1][0]=1; a[1][1]=-8; a[1][2]=3; b[1]=-4;	  //x_1-8x_2+3x_3-4=0
	a[2][0]=2; a[2][1]=1; a[2][2]=9; b[2]=12;	   //2x_1+x_2+9x_3+12=0
	
	int i;
	
	for(i=0; i<n; i++){	 
		x[i]=0;
	}
	cobisMethod(n, x, b, a);
	print(n, x);

	for(i=0; i<n; i++){	
		x[i]=0;
	}
	GaussSeidalMethod(n, x, b, a);
	print(n, x);
	
	return 0;
}

\end{minted}
\end{tcolorbox}
\vspace{\baselineskip}
\section*{Раздел 46.2: Нелинейное уравнение}
Уравнение типа \tcbox[nobeforeafter, tcbox raise base]{f(x) = \p{0}} является алгебраическим или трансцендентным. Эти
виды уравнений могут быть решены с использованием двух типов метода:
\begin{enumerate}
    \item \textbf{Прямой метод}: этот метод дает точное значение всех корней
непосредственно за конечное число шагов.
    \item \textbf{Косвенный или итерационный метод}: итерационный метод лучше всего
подходит для компьютерных программ для решения уравнения. Он основан
на концепции последовательного приближения. В итеративном методе есть
два способа решить уравнение:
    \begin{itemize}
        \item[$\circ$]\textbf{Метод скобок}: мы берем две начальные точки такие, что корень лежит между ними. Например- биективный метод, метод ложного позиционирования.
        \item[$\circ$]\textbf{Открытый метод}: мы берем одно или два начальных значения, что корень
может быть где угодно. Например - метода Ньютона-Рафсона, метод
последовательных приближений, метод секущих.
    \end{itemize}
\end{enumerate}
\vspace{\baselineskip}
Реализация в C:
\vspace{\baselineskip}
\vspace{\baselineskip}
\begin{tcolorbox}
\begin{minted}{C}
#define f(x) ( ((x)*(x)*(x)) - (x) - 2 )
#define f2(x) ( (3*(x)*(x)) - 1 )
#define g(x) ( cbrt( (x) + 2 ) )

double BisectionMethod(){
	double root=0;
	
	double a=1, b=2;
	double c=0;

	int loopCounter=0;
	if(f(a)*f(b)<0){
		while(1){
			loopCounter++;
			c=(a+b)/2;
			
			if(f(c)<0.00001 && f(c)>-0.00001){
				root=c;
				break;
			}

			if((f(a))*(f(c))<0){
				b=c;
			}else{
				a=c;
			}
		}
	}
	printf("It took %d loops.\n", loopCounter);
	
	return root;
}

double FalsePosition(){
	double root=0;
	
	double a=1, b=2;
	double c=0;
	
	int loopCounter=0;
	if(f(a)*f(b)<0){
		while(1){
			loopCounter++;
			
			c=(a*f(b)- b*f(a))/(f(b)- f(a));
			
			/*/printf("%lf\t %lf \n", c, f(c));/**////tes
			if(f(c)<0.00001 && f(c)>-0.00001){
				root=c;
				break;
			}

			if((f(a))*(f(c))<0){
				b=c;
			}else{
				a=c;
			}
		}
	}
	printf("It took %d loops.\n", loopCounter);
	
	return root;
}

double NewtonRaphson(){
	double root=0;
	
	double x1=1;
	double x2=0;
	
	int loopCounter=0;
	while(1){
		loopCounter++;
		
		x2 = x1 -(f(x1)/f2(x1));
		/*/printf("%lf \t %lf \n", x2, f(x2));/**////test
		
		if(f(x2)<0.00001 && f(x2)>-0.00001){
			root=x2;
			break;
		}
		
		x1=x2;
	}
	printf("It took %d loops.\n", loopCounter);
	
	return root;
}

double FixedPoint(){
	double root=0;
	double x=1;
	
	int loopCounter=0;
	while(1){
		loopCounter++;
		
		if((x-g(x))<0.00001 && (x-g(x))>-0.00001){
			root = x;
			break;
		}
		
		/*/printf("%lf \t %lf \n", g(x), x-(g(x)));/**////test
		
		x=g(x);
		
		}
		printf("It took %d loops.\n", loopCounter);
		
		return root;
	}

double Secant(){
	double root=0;
	
	double x0=1;
	double x1=2;
	double x2=0;
	
	int loopCounter=0;
	while(1){
			loopCounter++;
			
			/*/printf("%lf \t %lf \t %lf \n", x0, x1, f(x1));/**////test
			
			if(f(x1)<0.00001 && f(x1)>-0.00001){
				root=x1;
				break;
			}
			
			x2 =((x0*f(x1))-(x1*f(x0)))/(f(x1)-f(x0));
			
		   x0=x1;
		   x1=x2;
		}
		printf("It took %d loops.\n", loopCounter);
		
		return root;
	}
	
	int main(){
	double root;
	
	root = BisectionMethod();
	printf("Using Bisection Method the root is: %lf \n\n", root);
	
	root = FalsePosition();
	printf("Using False Position Method the root is: %lf \n\n", root);
	
	root = NewtonRaphson();
	printf("Using Newton-Raphson Method the root is: %lf \n\n", root);
	
	root = FixedPoint();
	printf("Using Fixed Point Method the root is: %lf \n\n", root);
	
	root = Secant();
	printf("Using Secant Method the root is: %lf \n\n", root);
	return 0;
}
\end{minted}
\end{tcolorbox}
\vspace{\baselineskip}

\newpage
\chapter*{Глава 47: Самая длинная общая подпоследовательность}
\section*{Раздел 47.1: Объяснение самой длинной общей подпоследовательности}
Одной из наиболее важных реализаций динамического программирования
является поиск\\ \href{https://en.wikipedia.org/wiki/Longest_common_subsequence_problem}{\underline{самой длинной общей подпоследовательности}}. Давайте
сначала определим некоторые основные термины.
\vspace{\baselineskip}

\textbf{Подпоследовательности}:
\vspace{\baselineskip}

Подпоследовательность - это последовательность, которая может быть
получена из другой последовательности путем удаления некоторых
символов без изменения порядка оставшихся символов. Допустим, у нас есть
строка \textbf{ABC}. Если мы сотрем ноль, один или более чем один символ из этой
строки, мы получаем подпоследовательность этой строки. Таким образом,
подпоследовательностями строки \textbf{ABC} будут \{"\textbf{A}", "\textbf{B}", "\textbf{C}", "\textbf{AB}", "\textbf{AC}", "\textbf{BC}", "\textbf{ABC}", "\textbf{ }"\}. Даже если мы удалим все символы, пустая строка также будет
подпоследовательностью. Чтобы обнаружить подпоследовательность, для
каждого символа в строке у нас есть два варианта - либо мы берем символ, либо нет.
Поэтому, если длина строки равна \textbf{n}, есть \textbf{2n} 
подпоследовательностей этой строки.
\vspace{\baselineskip}

\textbf{Самая длинная общая подпоследовательность:}
\vspace{\baselineskip}

Как следует из названия, из всех общих подпоследовательностей между
двумя строками самая длинная общая подпоследовательность (LCS) - та,
которая имеет максимальную длину. Например: общие
подпоследовательности между \textbf{"HELLOM"} и \textbf{"HMLD"} – это \textbf{"H"}, \textbf{"HL"}, \textbf{"HM"} и т.
д. Здесь \textbf{"HLL"} - самая длинная общая подпоследовательность, которая имеет
длину \textbf{3}.
\vspace{\baselineskip}

\textbf{Метод полного перебора}:
\vspace{\baselineskip}

Мы можем сгенерировать все подпоследовательности двух строк, используя
поиск с возвратом. Тогда мы можем сравнить их, чтобы узнать общие
подпоследовательности. После мы должны найти ту, что имеет
максимальную длину. Мы уже сталкивались с этим ранее, здесь \textbf{2n}
подпоследовательности строки длины \textbf{n}. Потребовались бы годы, чтобы
решить проблему, если бы наши \textbf{n} пересекали \textbf{20-25}.
\vspace{\baselineskip}

\textbf{Метод динамического программирования}:
\vspace{\baselineskip}

Давайте изучим данный метод на примере. Предположим, что у нас есть две
строки \textbf{abcdaf} и \textbf{acbcf}. Обозначим их \textbf{s1} и \textbf{s2}. Таким образом, самая длинная
общая подпоследовательность этих двух строк будет \textbf{"abcf"}, которая имеет
длину \textbf{4}. Напомню, что подпоследовательности не должны быть
непрерывными в строке. Чтобы построить \textbf{"abcf"}, мы проигнорировали \textbf{"da"}
в \textbf{s1} и \textbf{"c"} в \textbf{s2}. Как мы узнаем это с помощью динамического
программирования?
\vspace{\baselineskip}

Мы начнем с таблицы (двумерный массив), содержащей все символы \textbf{s1} в
строке и все символы \textbf{s2} в столбце. Здесь таблица имеет индекс \textbf{0}, и мы
помещаем символы от \textbf{1} и далее. Мы пройдем по таблице слева направо для
каждого ряда. Наша таблица будет выглядеть так:
\vspace{\baselineskip}
\begin{tcolorbox}
\begin{verbatim}
|     |     |  0  |  1  |  2  |  3  |  4  |  5  |  6  |
+-----+-----+-----+-----+-----+-----+-----+-----+-----+
|     | ch  |     |  a  |  b  |  c  |  d  |  a  |  f  |
+-----+-----+-----+-----+-----+-----+-----+-----+-----+
|  0  |     |     |     |     |     |     |     |     |
+-----+-----+-----+-----+-----+-----+-----+-----+-----+
|  1  |  a  |     |     |     |     |     |     |     |
+-----+-----+-----+-----+-----+-----+-----+-----+-----+
|  2  |  c  |     |     |     |     |     |     |     |
+-----+-----+-----+-----+-----+-----+-----+-----+-----+
|  3  |  b  |     |     |     |     |     |     |     |
+-----+-----+-----+-----+-----+-----+-----+-----+-----+
|  4  |  c  |     |     |     |     |     |     |     |
+-----+-----+-----+-----+-----+-----+-----+-----+-----+
|  5  |  f  |     |     |     |     |     |     |     |
+-----+-----+-----+-----+-----+-----+-----+-----+-----+
\end{verbatim}
\end{tcolorbox}
\vspace{\baselineskip}
Здесь каждая строка и столбец представляют длину самой длинной общей
подпоследовательности между двумя строками, если мы возьмем символы
этой строки и столбца и добавим к префиксу перед ним. Например: \textbf{Таблица[2][3]} представляет длину самой длинной общей подпоследовательности
между \textbf{"ac"} и \textbf{"abc"}.
0-й столбец представляет пустую подпоследовательность \textbf{s1}. Аналогично, 0-я
строка представляет пустую подпоследовательность \textbf{s2}. Если мы возьмем
пустую подпоследовательность строки и попытаемся сопоставить ее с другой
строкой, неважно, какова длина второй подстроки, общая
подпоследовательность будет иметь длину \textbf{0}. Таким образом, мы можем
заполнить 0-е строки и 0-е столбцы с 0. Мы получили:
\vspace{\baselineskip}
\begin{tcolorbox}
\begin{verbatim}
|     |     |  0  |  1  |  2  |  3  |  4  |  5  |  6  |
+-----+-----+-----+-----+-----+-----+-----+-----+-----+
|     | ch  |     |  a  |  b  |  c  |  d  |  a  |  f  |
+-----+-----+-----+-----+-----+-----+-----+-----+-----+
|  0  |     |  0  |  0  |  0  |  0  |  0  |  0  |  0  |
+-----+-----+-----+-----+-----+-----+-----+-----+-----+
|  1  |  a  |  0  |     |     |     |     |     |     |
+-----+-----+-----+-----+-----+-----+-----+-----+-----+
|  2  |  c  |  0  |     |     |     |     |     |     |
+-----+-----+-----+-----+-----+-----+-----+-----+-----+
|  3  |  b  |  0  |     |     |     |     |     |     |
+-----+-----+-----+-----+-----+-----+-----+-----+-----+
|  4  |  c  |  0  |     |     |     |     |     |     |
+-----+-----+-----+-----+-----+-----+-----+-----+-----+
|  5  |  f  |  0  |     |     |     |     |     |     |
+-----+-----+-----+-----+-----+-----+-----+-----+-----+
\end{verbatim}
\end{tcolorbox}
\vspace{\baselineskip}
Давайте начнем. Когда мы заполняем \textbf{таблицу[1][1]}, мы спрашиваем
себя,если бы у нас была только строка а и другая строка а, что будет самой
длинной общей подпоследовательностью? Длина LCS здесь будет \textbf{1}. Теперь
давайте посмотрим на \textbf{таблицу[1][2]}. У нас есть строка \textbf{ab} и строка \textbf{a}. Длина
LCS будет равна \textbf{1}. Как видите, остальные значения также будут равны \textbf{1} для
первой строки, так как она рассматривает только строку \textbf{a} с \textbf{abcd}, \textbf{abcda},
\textbf{abcdaf}. Так будет выглядеть наша таблица:
\vspace{\baselineskip}
\begin{tcolorbox}
\begin{verbatim}
|     |     |  0  |  1  |  2  |  3  |  4  |  5  |  6  |
+-----+-----+-----+-----+-----+-----+-----+-----+-----+
|     | ch  |     |  a  |  b  |  c  |  d  |  a  |  f  |
+-----+-----+-----+-----+-----+-----+-----+-----+-----+
|  0  |     |  0  |  0  |  0  |  0  |  0  |  0  |  0  |
+-----+-----+-----+-----+-----+-----+-----+-----+-----+
|  1  |  a  |  0  |  1  |  1  |  1  |  1  |  1  |  1  |
+-----+-----+-----+-----+-----+-----+-----+-----+-----+
|  2  |  c  |  0  |     |     |     |     |     |     |
+-----+-----+-----+-----+-----+-----+-----+-----+-----+
|  3  |  b  |  0  |     |     |     |     |     |     |
+-----+-----+-----+-----+-----+-----+-----+-----+-----+
|  4  |  c  |  0  |     |     |     |     |     |     |
+-----+-----+-----+-----+-----+-----+-----+-----+-----+
|  5  |  f  |  0  |     |     |     |     |     |     |
+-----+-----+-----+-----+-----+-----+-----+-----+-----+
\end{verbatim}
\end{tcolorbox}
\vspace{\baselineskip}
Для строки 2, которая теперь будет включать c. Для Таблицы [2] [1] у нас есть
ac с одной стороны и a с другой стороны. Так что длина LCS - 1. Откуда мы
взяли эту 1? Сверху, которая обозначает что LCS a между двумя подстроками.

И что мы говорим, что если \textbf{s1[2]} и \textbf{s2[1]} не совпадают, то длина LCS будет
максимальной из длины LCS сверху или слева. Взятие длины LCS сверху означает, что мы не берем
текущий символ из \textbf{s2}. Точно также, принимая длину LCS слева, мы не берем
текущий символ из \textbf{s1} для создания LCS. Мы получили:
\vspace{\baselineskip}
\begin{tcolorbox}
\begin{verbatim}
|     |     |  0  |  1  |  2  |  3  |  4  |  5  |  6  |
+-----+-----+-----+-----+-----+-----+-----+-----+-----+
|     | ch  |     |  a  |  b  |  c  |  d  |  a  |  f  |
+-----+-----+-----+-----+-----+-----+-----+-----+-----+
|  0  |     |  0  |  0  |  0  |  0  |  0  |  0  |  0  |
+-----+-----+-----+-----+-----+-----+-----+-----+-----+
|  1  |  a  |  0  |  1  |  1  |  1  |  1  |  1  |  1  |
+-----+-----+-----+-----+-----+-----+-----+-----+-----+
|  2  |  c  |  0  |  1  |     |     |     |     |     |
+-----+-----+-----+-----+-----+-----+-----+-----+-----+
|  3  |  b  |  0  |     |     |     |     |     |     |
+-----+-----+-----+-----+-----+-----+-----+-----+-----+
|  4  |  c  |  0  |     |     |     |     |     |     |
+-----+-----+-----+-----+-----+-----+-----+-----+-----+
|  5  |  f  |  0  |     |     |     |     |     |     |
+-----+-----+-----+-----+-----+-----+-----+-----+-----+
\end{verbatim}
\end{tcolorbox}
\vspace{\baselineskip}
Итак, наша первая формула будет:
\vspace{\baselineskip}
\begin{tcolorbox}
\begin{minted}{C}     
if s2[i] is not equal to s1[j]
    Table[i][j]= max(Table[i-1][j], Table[i][j-1]
endif
\end{minted}
\end{tcolorbox}
\vspace{\baselineskip}
Двигаясь дальше, для \textbf{Таблицы[2][2]} мы имеем строку \textbf{ab} и \textbf{ac}. Поскольку \textbf{c} и
\textbf{b} не совпадают, мы ставим максимум вершины или оставляем здесь. В
данном случае это снова \textbf{1}. После этого для \textbf{Таблицы[2][3]} у нас есть строки
\textbf{abc} и \textbf{ac}. На этот раз текущие значения и строки и столбца одинаковы. Теперь
длина LCS будет равна максимальной длине LCS + 1. Как мы можем получить
максимальную длину LCS до сих пор? Мы проверяем диагональное
значение, которое представляет лучшее соответствие между \textbf{ab} и \textbf{a}. Из этого
состояния для текущих значений мы добавляем еще один символ к \textbf{s1} и \textbf{s2}, с
которыми случилось то же самое. Так что длина LCS, конечно, увеличится.
Мы положим \textbf{1 + 1 = 2} в \textbf{таблице[2][3]}. Мы получили:
\vspace{\baselineskip}
\begin{tcolorbox}
\begin{verbatim}
|     |     |  0  |  1  |  2  |  3  |  4  |  5  |  6  |
+-----+-----+-----+-----+-----+-----+-----+-----+-----+
|     | ch  |     |  a  |  b  |  c  |  d  |  a  |  f  |
+-----+-----+-----+-----+-----+-----+-----+-----+-----+
|  0  |     |  0  |  0  |  0  |  0  |  0  |  0  |  0  |
+-----+-----+-----+-----+-----+-----+-----+-----+-----+
|  1  |  a  |  0  |  1  |  1  |  1  |  1  |  1  |  1  |
+-----+-----+-----+-----+-----+-----+-----+-----+-----+
|  2  |  c  |  0  |  1  |  1  |  2  |     |     |     |
+-----+-----+-----+-----+-----+-----+-----+-----+-----+
|  3  |  b  |  0  |     |     |     |     |     |     |
+-----+-----+-----+-----+-----+-----+-----+-----+-----+
|  4  |  c  |  0  |     |     |     |     |     |     |
+-----+-----+-----+-----+-----+-----+-----+-----+-----+
|  5  |  f  |  0  |     |     |     |     |     |     |
+-----+-----+-----+-----+-----+-----+-----+-----+-----+
\end{verbatim}
\end{tcolorbox}
\vspace{\baselineskip}
Итак, наша вторая формула будет:
\begin{tcolorbox}
\begin{minted}{C}     
if s2[i] equals to s1[j]
    Table[i][j]= Table[i-1][j-1]+1
endif
\end{minted}
\end{tcolorbox}
\vspace{\baselineskip}
Мы определили оба случая. Используя эти две формулы, мы можем
заполнить всю таблицу. После заполнения таблица будет выглядеть так:
\vspace{\baselineskip}
\begin{tcolorbox}
\begin{verbatim}
|     |     |  0  |  1  |  2  |  3  |  4  |  5  |  6  |
+-----+-----+-----+-----+-----+-----+-----+-----+-----+
|     | ch  |     |  a  |  b  |  c  |  d  |  a  |  f  |
+-----+-----+-----+-----+-----+-----+-----+-----+-----+
|  0  |     |  0  |  0  |  0  |  0  |  0  |  0  |  0  |
+-----+-----+-----+-----+-----+-----+-----+-----+-----+
|  1  |  a  |  0  |  1  |  1  |  1  |  1  |  1  |  1  |
+-----+-----+-----+-----+-----+-----+-----+-----+-----+
|  2  |  c  |  0  |  1  |  1  |  2  |  2  |  2  |  2  |
+-----+-----+-----+-----+-----+-----+-----+-----+-----+
|  3  |  b  |  0  |  1  |  2  |  2  |  2  |  2  |  2  |
+-----+-----+-----+-----+-----+-----+-----+-----+-----+
|  4  |  c  |  0  |  1  |  2  |  3  |  3  |  3  |  3  |
+-----+-----+-----+-----+-----+-----+-----+-----+-----+
|  5  |  f  |  0  |  1  |  2  |  3  |  3  |  3  |  4  |
+-----+-----+-----+-----+-----+-----+-----+-----+-----+
\end{verbatim}
\end{tcolorbox}
\vspace{\baselineskip}
Длина LCS между \textbf{s1} и \textbf{s2} будет \textbf{Таблицей[5][6] = 4}. Здесь 5 и 6 - длина \textbf{s2} и \textbf{s1}
соответственно. Наш псевдокод будет выглядеть так:
\vspace{\baselineskip}
\begin{tcolorbox}
\begin{minted}{C}
Procedure LCSlength(s1, s2):
Table[0][0]=0
for i from 1 to s1.length
	Table[0][i]=0
endfor
for i from 1 to s2.length
	Table[i][0]=0
endfor
for i from 1 to s2.length
	for j from 1 to s1.length
		if s2[i] equals to s1[j]
			Table[i][j]= Table[i-1][j-1]+1
		else
			Table[i][j]= max(Table[i-1][j], Table[i][j-1])
		endif
	endfor
endfor
Return Table[s2.length][s1.length]
\end{minted}
\end{tcolorbox}
\vspace{\baselineskip}
Временная сложность для этого алгоритма: \textbf{O (mn)}, где \textbf{m} и \textbf{n} обозначают
длину каждой строки.
\vspace{\baselineskip}

Как мы узнаем самую длинную общую подпоследовательность? Начнем с
правого нижнего угла. Мы будем проверять, откуда приходит значение. Если
значение приходит из диагонали, то есть если \textbf{Таблица[i-1][j-1]} равна
\textbf{Таблице[i][j] - 1}, мы вставляем либо \textbf{s2[i]}, либо \textbf{s1[j]} (оба одинаковые) и
движемся по диагонали. Если значение приходит сверху, это означает, что
если \textbf{Таблица[i-1][j]} равна \textbf{Таблице[i][j]}, мы переходим к вершине. Если значение
идет слева, значит, что если Таблица \textbf{[i][j-1]} равна \textbf{Таблице[i][j]}, мы
двигаемся влево. Когда мы достигаем крайнего левого или верхнего столбца,
наш поиск заканчивается. Затем мы извлекаем значения из стека и печатаем
их. Псевдокод:
\vspace{\baselineskip}
\begin{tcolorbox}
\begin{minted}{C}
Procedure PrintLCS(LCSlength, s1, s2)
temp := LCSlength
S = stack()
i := s2.length
j := s1.length
while i is not equal to 0 and j is not equal to 0
		if Table[i-1][j-1]== Table[i][j]-1 and s1[j]==s2[i]
		S.push(s1[j]) 
		i := i -1		
		j := j -1
	elseif Table[i-1][j]== Table[i][j]
		i := i-1
	else
		j := j-1
   endif
endwhile
while S is not empty
	print(S.pop)
endwhile
\end{minted}
\end{tcolorbox}
\vspace{\baselineskip}
Следует отметить: если \textbf{Таблица[i-1][j]} и \textbf{Таблица[i][j-1]} равны \textbf{Таблице[i][j]}
и \textbf{Таблица[i-1][j-1]} не равна \textbf{Таблице [i][j] - 1}, то может быть две LCS. Этот
псевдокод не учитывает такую ситуацию. Вы должны будете решить эту
проблему рекурсивно, чтобы найти несколько LCS
\vspace{\baselineskip}
.
Временная сложность для этого алгоритма: \textbf{O (max (m, n))}.

\newpage
\chapter*{Глава 48: Cамая длинная возрастающая подпоследовательность}
\section*{Раздел 48.1: Самая длинная возрастающая подпоследовательность.}
Проблема \href{https://en.wikipedia.org/wiki/Longest_increasing_subsequence}{\underline{самой длинной возрастающей подпоследовательности}} состоит в
том, чтобы найти подпоследовательность из заданной входной
последовательности, в которой элементы подпоследовательности
сортируются в порядке убывания. Все подпоследовательности не являются
смежными или уникальными.
\vspace{\baselineskip}

\textbf{Применение самой длинной возрастающей подпоследовательности}:
\vspace{\baselineskip}

Такие алгоритмы как: самая длинная возрастающая подпоследовательность,
самая длинная общая подпоследовательность используются в управлении
версиями таких систем как Git и т. д.
\vspace{\baselineskip}

\textbf{Простая форма алгоритма:}
\vspace{\baselineskip}
\begin{enumerate}
    \item Найти уникальные строки, которые являются общими для обоих
документов.
    \item Взять все такие строки из первого документа и упорядочить их в
соответствии с их появлением во втором документе.
    \item Вычислить LIS полученной последовательности (выполнив Терпеливую
сортировку), получив самую длинную подходящую последовательность
строк - соответствие между строками двух документов.
    \item Повторяйте алгоритм на каждом диапазоне линий между уже
подобранными.
\end{enumerate}
Теперь давайте рассмотрим более простой пример проблемы LCS. Здесь
вход представляет собой только одну последовательность различных целых
чисел \tcbox[nobeforeafter, tcbox raise base]{a1, a2, ..., an.}, и мы хотим найти самую длинную возрастающую
подпоследовательность в ней. Например, если входным значением будут
\textbf{7,3,8,4,2,6} тогда самая длинная возрастающая подпоследовательность
составляет \textbf{3,4,6}.
\vspace{\baselineskip}

Самый простой подход заключается в сортировке входных элементов в
порядке возрастания и применении алгоритма LCS к исходной и
отсортированной последовательности. Однако, если вы посмотрите на
результирующий массив, вы заметите, что многие значения одинаковы, и
массив выглядит повторяющимся. Это говорит о том, что проблема LIS (самая
длинная возрастающая подпоследовательность) может быть решена с
применением алгоритма динамического программирования, использующим
только одномерный массив.
\vspace{\baselineskip}

\textbf{Псевдокод}:
\begin{enumerate}
    \item Описать массив значений, которые мы хотим вычислить.\\
Для \tcbox[nobeforeafter, tcbox raise base]{\p{1} <= i <= n} пусть \textbf{A(i)} будет длиной самой длинной увеличивающейся
последовательности ввода. Обратите внимание, что длина ,которая в
+конечном итоге нас\\ интересует - \tcbox[nobeforeafter, tcbox raise base]{ max \{A (i) | \p{1} $\leq$ i $\leq$ n\} }.
    \item Задать рекуррентность.\\
Для \tcbox[nobeforeafter, tcbox raise base]{\p{1}<= i <= n, A(i)=\p{1}+ max\{A(j)|\p{1} $\leq$ j < i and input(j)< input(i)\}}.
    \item Вычислить значения А.
    \item Найти оптимальное решение.
\end{enumerate}
\vspace{\baselineskip}
Следующая программа использует A для вычисления оптимального
решения. Первая часть вычисляет значение m так, что \textbf{A(m)} - длина
оптимальной возрастающей подпоследовательности ввода. Вторая часть
вычисляет оптимальное увеличение подпоследовательности, для удобства
выводим её в обратном порядке. Эта программа выполняется за время O(n),
поэтому весь алгоритм работает за время O($n^2$).
\vspace{\baselineskip}

\textbf{Часть 1:}
\vspace{\baselineskip}
\begin{tcolorbox}
\begin{minted}{C}     
m <- 1
for i :2..n
	if A(i) > A(m) then
		m <- i	
	end if
end for
\end{minted}
\end{tcolorbox}
\vspace{\baselineskip}
\textbf{Часть 2:}
\vspace{\baselineskip}
\begin{tcolorbox}
\begin{minted}{C}
put a
while A(m) > 1 do
	i <- m - 1
	while not(ai < am and A(i)= A(m) - 1) do
	i <- i - 1
	end while
	m <- i
	put a 
end while
\end{minted}
\end{tcolorbox}
\vspace{\baselineskip}
\textbf{Рекурсивное решение:}
\vspace{\baselineskip}

\textbf{Подход 1:}
\vspace{\baselineskip}
\begin{tcolorbox}
\begin{minted}{C}
LIS(A[1..n]):
	if(n =0) then return 0
	m = LIS(A[1..(n - 1)])
	B is subsequence of A[1..(n - 1)] with only elements less than a[n]
	(* let h be size of B, h <= n-1*)
	m = max(m, 1+ LIS(B[1..h]))
	Output m
\end{minted}
\end{tcolorbox}
\vspace{\baselineskip}
\textbf{Временная сложность в подходе 1:} O($n^2$)
\vspace{\baselineskip}

\textbf{Подход 2:}
\vspace{\baselineskip}
\begin{tcolorbox}
\begin{minted}{C}
LIS(A[1..n], x):
	if(n = 0) then return 0
	m = LIS(A[1..(n - 1)], x)
	if(A[n]< x) then
		m = max(m, 1+ LIS(A[1..(n - 1)], A[n]))
	Output m
	
MAIN(A[1..n]):
	return LIS(A[1..n], infinity)
\end{minted}
\end{tcolorbox}
\vspace{\baselineskip}

\textbf{Временная сложность в подходе 2:} O($n^2$)
\vspace{\baselineskip}

\textbf{Подход 3:}
\vspace{\baselineskip}
\begin{minted}{C}
LIS(A[1..n]):
	if(n = 0) return 0
	m = 1 
	for i = 1 to n-1 do
		if(A[i]< A[n]) then
			m = max(m, 1+ LIS(A[1..i]))
	return m
MAIN(A[1..n]):
	return LIS(A[1..i])
\end{minted}
\vspace{\baselineskip}

\textbf{Временная сложность в подходе 3:} O($n^2$)
\vspace{\baselineskip}

\textbf{Итерационный алгоритм:}
\vspace{\baselineskip}

Вычисляет значения итеративно снизу вверх.
\vspace{\baselineskip}
\begin{tcolorbox}
\begin{minted}{C}
LIS(A[1..n]):
	Array L[1..n]
	(* L[i]= value of LIS ending(A[1..i])*)
	for i =1 to n do
		L[i]=1
		for j =1 to i-1 do
			if(A[j]< A[i]) do
				L[i]= max(L[i], 1+ L[j])
	return L
	
MAIN(A[1..n]):
	L = LIS(A[1..n])
		return the maximum value in L
\end{minted}
\end{tcolorbox}
\vspace{\baselineskip}

\textbf{Временная сложность в итерационном подходе:} O($n^2$)
\vspace{\baselineskip}

\textbf{Вспомогательное пространство:} O(n)
\vspace{\baselineskip}

\vspace{\baselineskip}

\vspace{\baselineskip}

Давайте возьмем \textbf{\{0, 8, 4, 12, 2, 10, 6, 14, 1, 9, 5, 13, 3, 11, 7, 15\}} в качестве
входных данных. Итак, самая длинная возрастающая подпоследовательность
для данного входа - \textbf{\{0, 2, 6, 9, 11, 15\}}.

\newpage
\chapter*{Глава 49: Проверка двух строк на анаграмму}
Две строки с одинаковым набором символов называются анаграммой. Здесь
я использовал javascript.
\vspace{\baselineskip}

Мы создадим хэш str1 и увеличим количество + 1. Мы сделаем цикл на 2-й
строке и проверим, все ли символы есть в хэше, и уменьшим значения хэш-
ключа. Проверьте, если все значения хэш-ключа равны нулю, тогда будет
анаграмма.
\section*{Раздел 49.1: Ввод и вывод образцов}
Пример 1:
\vspace{\baselineskip}
\begin{tcolorbox}
\begin{minted}{Python}     
let str1 ='stackoverflow';
let str2 ='flowerovstack';
\end{minted}
\end{tcolorbox}
\vspace{\baselineskip}
Эти строки - анаграммы.
\vspace{\baselineskip}

// Создайте хэш из str1 и увеличьте один счетчик.
\vspace{\baselineskip}
\begin{tcolorbox}
\begin{minted}{C}     
hashMap ={
    s :1,    
    t :1,    
    a :1,    
    c :1,   
    k :1,
    o :2,
    v :1,
    e :1,
    r :1,
    f :1,
    l :1, 
    w :1
}
\end{minted}
\end{tcolorbox}
\vspace{\baselineskip}
Вы можете видеть, что хэш-ключ 'o' содержит значение 2, потому что o
встречается 2 раза в строке.
\vspace{\baselineskip}

Теперь зациклите str2 и проверьте, чтобы каждый символ присутствовал в
hashMap, если да, уменьшите значение ключа hashMap, иначе верните
Ложь(false) (что указывает на то, что это не анаграмма).
\vspace{\baselineskip}
\begin{tcolorbox}
\begin{minted}{C}     
hashMap ={
    s :0,
    t :0, 
    a :0,  
    c :0,  
    k :0,  
    o :0,  
    v :0, 
    e :0,  
    r :0,   
    f :0,   
    l :0,  
    w :0
}
\end{minted}
\end{tcolorbox}
\vspace{\baselineskip}
Теперь зациклите объект hashMap и проверьте, что все значения равны нулю
в ключе hashMap.
\vspace{\baselineskip}

В нашем случае все значения равны нулю, так что это анаграмма.
\section*{Раздел 49.2: Общий код для анаграмм}
\vspace{\baselineskip}
\begin{tcolorbox}
\begin{minted}{C}     
(function(){

	var hashMap ={};
	
	function isAnagram (str1, str2){
	
		if(str1.length!== str2.length){
			return false;
		}
		
		value one (+1).
		createStr1HashMap(str1);

		var valueExist = createStr2HashMap(str2);

		return isStringsAnagram(valueExist);
	}
	
	function createStr1HashMap (str1){
		[].map.call(str1, function(value, index, array){
			hashMap[value]= value in hashMap ?(hashMap[value]+1):1;
			return value;
		});
	}
	
	function createStr2HashMap (str2){
		var valueExist =[].every.call(str2, function(value, index, array){
			if(value in hashMap){
				hashMap[value]= hashMap[value]-1;
			}
			return value in hashMap;
		});
		return valueExist;
	}
	
	function isStringsAnagram (valueExist){
		if(!valueExist){
			return valueExist;
			}else{
				var isAnagram;
				for(var i in hashMap){
					if(hashMap[i]!==0){
						isAnagram =false;
						break;
						}else{
							isAnagram =true;
						}
					}
					return isAnagram;
				}
		}
		
	isAnagram("stackoverflow", "flowerovstack");
	isAnagram("stackoverflow", "flowervvstack");

\end{minted}
\end{tcolorbox}

\vspace{\baselineskip}
Временная сложность: 3n, т.е. O(n).