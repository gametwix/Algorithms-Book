% Назарова

\chapter*{Глава 50 : Треугольник Паскаля}
\section*{Раздел 50.1: Треугольник Паскаля на C}
\begin{tcolorbox}\begin{minted}{C}
int i, space, rows, k=0, count = 0, count1 = 0; /*rows - кол-во строк, 
//space - кол-во отступов, count - кол-во чисел до "середины строки",
//k (count1) - число, на которое числа увеличиваются(уменьшаются)*/
rows=5;

//вывод треугольника Паскаля
 for(i=1; i<=rows; ++i)//подсчет выведенных строк
 {
	for(space=1; space <= rows-i; ++space)/*создание отступов для выравнивания
	//треугольника. количество отступов соответствует (число строк - номер 
	//строки)*/
	{
		printf("  ");
		++count;
	}
	while(k != 2*i-1)//количество чисел в строке
	{
		if (count <= rows-1)/*вывод первой половины строки. первое число - 
		//номер строки, 
		//последующие - увеличиваются на единицу. 
		//максимальное число - середина строки - с номером rows по счету*/
		{
			printf("%d ", i+k);
			++count;
		}
		else/*вывод второй половины строки. каждое последующее число уменьшается 
		//на единицу*/
		{
			++count1;
			printf("%d ", (i+k-2*count1));
		}
		++k;
	}
	count1 = count = k = 0;
	printf("\n");//переход к новой строке
 }

\end{minted}
\end{tcolorbox}
\vspace{\baselineskip}

\textbf{Выход}
\vspace{\baselineskip}
\begin{tcolorbox}
{\color{Purple}{
\hspace{14,5mm}1 \newline
\hspace*{11mm}2 3 2\newline
\hspace*{7,5mm}3 4 5 4 3\newline
\hspace*{4,5mm}4 5 6 7 6 5 4\newline
\hspace*{1mm}5 6 7 8 9 8 7 6 5
}}

\end{tcolorbox}

\chapter*{Глава 51: Алгоритм:- Вывод матрицы m*n по спирали}
Примеры ввода и вывода приведены ниже.
\section*{Раздел 51.1: Пример}
\begin{tcolorbox}
               Input:
               
\vspace{\baselineskip}
14 15 16 17 18 21\newline
19 10 20 11 54 36\newline
64 55 44 23 80 39\newline 
91 92 93 94 95 42

\vspace{\baselineskip}
Output:\newline 
print value in index\newline
14 15 16 17 18 21 36 39 42 95 94 93 92 91 64 19 10 20 11 54 80 23 44 55

\vspace{\baselineskip}
or print index\newline
00 01 02 03 04 05 15 25 35 34 33 32 31 30 20 10 11 12 13 14 24 23 22 21 
\end{tcolorbox}
\section*{Раздел 51.2: Напишем общий код}
\begin{tcolorbox}\begin{minted}{Python}
#число "уровней" спирали
function noOfLooping(m,n) {
	# ищем минимальную сторону
	if(m > n) {
		smallestValue = n;
	} else {
		smallestValue = m;
	}
	# делим ее пополам с округлением вверх
	if(smallestValue % 2 == 0) {
			return smallestValue/2;
	} else {
			return (smallestValue+1)/2;
	}
 }
#вывод матрицы по спирали
function squarePrint(m,n) {
	var looping = noOfLooping(m,n);#количество повторений

	#вывод одного "уровня" спирали
	for(var i = 0; i < looping; i++) {
		for(var j = i; j < m - 1 - i; j++) {#верхняя горизонталь
				console.log(i+''+j);#вывести индексы i, j
		}
		for(var k = i; k < n - 1 - i; k++) {#правая вертикаль
				console.log(k+''+j);
		}
		for(var l = j; l > i; l--) {#нижняя горизонталь
				console.log(k+''+l);
		}
		for(var x = k; x > i; x--) {#левая вертикаль
				console.log(x+''+l);
		}
	}
 }
squarePrint(6,4);#применить к матрице размером 4x6

\end{minted}
\end{tcolorbox}
\chapter*{Глава 52: Возведение матрицы в степень}
\section*{Раздел 52.1: Возведение матрицы в степень для решения типовых задач}

\textit{Найти f(n): n-е число Фибоначчи.} Задача довольно проста, когда \textbf{n} относительно маленькое. Мы можем использовать простую рекурсию, f(n) = f(n-\p{1}) + f(n-\p{2}), или динамическое программирование, чтобы избежать вычисления одной и той же функции снова и снова. Но что вы будете делать, если в задаче сказано: \textbf{при 0 < n < $10^9$ найти f (n) mod 999983?} Динамическое программирование не подойдет, так как же мы решим эту задачу?
\vspace{\baselineskip}

Сначала давайте посмотрим, как возведение матрицы в степень может помочь представить рекурсивное отношение.
\vspace{\baselineskip}

\textbf{Необходимые умения:}
\begin{itemize}
    \itemИмея две матрицы, необходимо знать, как найти их произведение. Далее, имея матрицу, полученную путем перемножения этих двух матриц, и одну из этих матриц, необходимо знать, как найти другую матрицу.
    \item Имея матрицу размера \textbf{d $\times$ d}, необходимо знать, как найти ее n-ю степень за \textbf{O(d3log (n))}.
\end{itemize}
\textbf{Шаблоны:}

\vspace{\baselineskip}
Сначала нам нужно рекурсивное отношение, а после  мы должны найти матрицу \textbf{M}, которая может привести нас к желаемому состоянию из множества уже известных состояний. Предположим, что мы знаем \textbf{k} состояний данного рекуррентного отношения и хотим найти состояние \textbf{(k + 1)}. Пусть \textbf{M} - матрица размера \textbf{k $\times$ k}, и из известных состояний рекуррентного отношения мы строим матрицу \textbf{A:[k $\times$ 1]}, после мы строим матрицу \textbf{B:[k $\times$ 1]}, которая будет представлять множество следующих состояний, т. е. \textbf{M $\times$ A = B}, как показано ниже:
\begin{tcolorbox}
  
\hspace{12mm} |\hspace{3,5mm}f(n)\hspace{4,4mm}|\hspace{12,5mm}|\hspace{2,5mm}f(n+\p{1})\hspace{3mm}|\newline
\hspace*{13,3mm}|\hspace{1mm} f(n-\p{1})\hspace{1mm} |\hspace{12,2mm}|\hspace{4mm}f(n)\hspace{7mm}|\newline
\hspace*{3,5mm}M $\times$ |\hspace*{1mm} f(n-\p{2}) \hspace*{1mm}|\hspace*{2,5mm}  = \hspace*{2,5mm} |\hspace*{3,2mm}f(n-\p{1})\hspace*{2,8mm} |\newline
\hspace*{13,3mm}|\hspace*{3mm} ...... \hspace*{2,4mm}|  \hspace*{9,5mm}   |\hspace*{3mm} ...... \hspace*{3,7mm} |\newline
\hspace*{13,3mm}|\hspace*{1,1mm} f(n-k)\hspace*{1,1mm} | \hspace*{9,3mm} |\hspace*{1,1mm}f(n-k+\p{1})\hspace*{1,1mm}|

\end{tcolorbox}

Так что если мы сможем построить \textbf{M} в соответствии с вышесказанным, то наша задача будет выполнена! Затем матрица будет использоваться для представления рекуррентного отношения.

\vspace{\baselineskip}
\textbf{Тип 1:}


Начнем с самого простого,f(n) = f(n-\p{1}) + f(n-\p{2}) .\newline Мы получаем, f(n+\p{1}) = f(n) + f(n-\p{1}).\newline
Предположим, что мы знаем f(n) и f(n-\p{1}); Мы хотим найти f(n+\p{1})\newline
Из ситуации, описанной выше, матрица \textbf{A} и матрица \textbf{В} могут быть построены так, как показано ниже:

\vspace{\baselineskip}
\begin{tcolorbox}
\hspace{3,5mm} Matrix A \hspace{20mm}Matrix B \newline
\hspace*{3mm}|\hspace{5mm}  f(n)  \hspace*{5mm}|  \hspace*{13mm} |\hspace*{2mm} f(n+\p{1})\hspace*{2mm} |\newline
\hspace*{3mm}|\hspace*{3mm} f(n-\p{1})\hspace*{3,5mm} |  \hspace*{13mm} |\hspace*{4mm}  f(n) \hspace{4mm}  |
\end{tcolorbox}

\vspace{\baselineskip}
[Примечание: Матрица \textbf{A} всегда будет построена таким образом, что каждое состояние, от которого зависит f(n+1) , будет присутствовать]
Теперь нам нужно построить матрицу \textbf{M} размера \textbf{2$\times$2} таким образом, чтобы она удовлетворяла \textbf{M $\times$ A = B}, как указано выше.\newline
Первый элемент \textbf{B} - это f(n+\p{1}), что на самом деле является f(n) + f(n-\p{1}). Чтобы получить это из матрицы \textbf{A}, нам нужно \textbf{1 $\times$ f(n)} и \textbf{1 $\times$ f (n-1)}. Таким образом, первая строка \textbf{M} будет равна \textbf{[1 1]}.

\vspace{\baselineskip}
\begin{tcolorbox}
|\hspace{2mm} \p{1}\hspace{6mm}\p{1}\hspace{2mm}|\hspace{3mm} $\times$ \hspace{3mm}| \hspace{3mm} f(n) \hspace{3mm}|\hspace{3mm} = \hspace{3mm} |\hspace{1,5mm} f(n+\p{1})\hspace{1,5mm} |\newline
|\hspace*{0,3mm} - - - - -\hspace{0,3mm} |\hspace*{10,5mm}     |\hspace*{2,5mm} f(n-\p{1})\hspace*{2,7mm}| \hspace*{10,5mm}    | - - - - - - |
\end{tcolorbox}

\vspace{\baselineskip}
[Примечание: - - - - - означает, что нам не важно это значение.]\newline
Аналогично, 2-й элемент \textbf{B} это f(n), который можно получить просто взяв \textbf{1 $\times$ f (n)} из \textbf{A}, поэтому 2-я строка \textbf{M} равна [1 0].

\vspace{\baselineskip}
\begin{tcolorbox}
|\hspace{0,3mm} - - - - - \hspace{0,3mm}|\hspace{3mm} $\times$ \hspace{3mm}| \hspace{3mm} f(n) \hspace{3mm}|\hspace{3mm}  = \hspace{3mm} |\hspace{0,3mm} - - - - - \hspace{0,3mm} |\newline
|\hspace{1,5mm} \p{1}\hspace{6mm}\p{0}\hspace{1,5mm} |\hspace*{10,5mm}     |\hspace*{2,5mm} f(n-\p{1})\hspace*{2,9mm}| \hspace*{10,5mm}    |\hspace*{4,5mm} f(n)\hspace*{4,5mm}|
\end{tcolorbox}

\vspace{\baselineskip}
Тогда мы получим нашу желаемую матрицу \textbf{M} размера \textbf{2 $\times$ 2}.

\vspace{\baselineskip}
\begin{tcolorbox}
|\hspace{1,5mm} \p{1}\hspace{6mm}\p{1}\hspace{1,5mm}|\hspace{3mm} $\times$ \hspace{3mm}| \hspace{3mm} f(n) \hspace{3mm}|\hspace{3mm}  = \hspace{3mm} |\hspace*{2,5mm} f(n+\p{1})\hspace*{2,9mm} |\newline
|\hspace{1,5mm} \p{1}\hspace{6mm}\p{0}\hspace{0,1mm} |\hspace*{10,7mm}     |\hspace*{2,5mm} f(n-\p{1})\hspace*{2,9mm}| \hspace*{10,5mm}    |\hspace*{6mm} f(n)\hspace*{6mm}|
\end{tcolorbox}

\vspace{\baselineskip}
Эти матрицы просто выводятся с помощью матричного умножения.

\vspace{\baselineskip}

\textbf{Тип 2:}

\vspace{\baselineskip}
Давайте немного усложним задачу: найдем f(n) = a $\times$ f(n-\p{1}) + b $\times$ f(n-\p{2}), где \textbf{a} и \textbf{b} - константы.\newline
Из этого следует, что f(n+\p{1}) = a $\times$ f(n) + b $\times$ f(n-\p{1}).\newline 
К этому моменту должно быть ясно, что размерность матриц будет равна числу зависимостей, т. е. в данном примере снова 2. Таким образом, для \textbf{A} и \textbf{B} мы можем построить две матрицы размера \textbf{2 $\times$ 1}:

\vspace{\baselineskip}
\begin{tcolorbox}
\hspace{3,5mm} Matrix A \hspace{20mm}Matrix B \newline
\hspace*{3mm}|\hspace{5mm}  f(n)  \hspace*{5mm}|  \hspace*{13mm} |\hspace*{2mm} f(n+\p{1})\hspace*{2mm} |\newline
\hspace*{3mm}|\hspace*{3mm} f(n-\p{1})\hspace*{3,5mm} |  \hspace*{13mm} |\hspace*{4mm}  f(n) \hspace{4mm}  |
\end{tcolorbox}

\vspace{\baselineskip}
Теперь для f(n+\p{1}) = a $\times$ f(n) + b $\times$ f(n-\p{1}) нам нужно иметь [a, b] в первой строке искомой матрицы \textbf{M}. Что касается 2-го элемента в \textbf{B}, т. е. f(n), мы уже знаем его из матрицы \textbf{A}, поэтому мы просто возьмем значение оттуда, из чего следует, что 2-я строка матрицы M равна [1 0]. На этот раз мы получим:

\vspace{\baselineskip}
\begin{tcolorbox}
|\hspace{1,5mm} a \hspace{3mm}  b \hspace{0,1mm}|\hspace{3mm} $\times$ \hspace{3mm}| \hspace{3mm} f(n) \hspace{3mm}|\hspace{3mm}  = \hspace{3mm} |\hspace*{2,5mm} f(n+\p{1})\hspace*{2,9mm} |\newline
|\hspace{1,5mm} \p{1}\hspace{6mm}\p{0}\hspace{0,1mm} |\hspace*{10,7mm}     |\hspace*{2,5mm} f(n-\p{1})\hspace*{2,9mm}| \hspace*{10,5mm}    |\hspace*{6mm} f(n)\hspace*{6mm}|
\end{tcolorbox}


\vspace{\baselineskip}
Довольно просто, да?

\vspace{\baselineskip}
\textbf{Тип 3:}

\vspace{\baselineskip}
Если вы дошли до этого этапа, вы явно повзрослели и готовы столкнуться с немного сложным соотношением: найти f(n) = a $\times$ f(n-\p{1}) + c $\times$ f(n-\p{3})?\newline 
Упс! Несколько минут назад все, что мы видели, имело непрерывные состояния, но здесь состояние \textbf{f(n-2)} отсутствует. Что теперь?

\vspace{\baselineskip}
На самом деле это уже не проблема, мы можем преобразовать соотношение следующим образом: f(n) = a $\times$ f(n-\p{1}) + \p{0} $\times$ f(n-\p{2}) + c $\times$ f(n-\p{3}), получая f(n+\p{1}) = a $\times$ f(n) + \p{0} $\times$ f(n-\p{1}) + c $\times$ f(n-\p{2}). Теперь мы видим, что это на самом деле форма, описанная в типе 2. Таким образом, здесь искомая матрица \textbf{M} будет иметь размер \textbf{3 X 3}, а ее элементы равны:

\vspace{\baselineskip}
\begin{tcolorbox}
|\hspace{1,5mm} a \hspace{3mm}  \p{0} \hspace{3mm} c \hspace{0,5mm}| \hspace{11mm}| \hspace{3mm} f(n) \hspace{3mm}|\hspace{16mm} |\hspace*{2,5mm} f(n+\p{1})\hspace*{2,9mm} |\newline
|\hspace{1,5mm} \p{1}\hspace{6mm}\p{0}\hspace{6mm}\p{0}\hspace{0mm} |\hspace*{3mm} $\times$\hspace{3mm}     |\hspace*{2,5mm} f(n-\p{1})\hspace*{2,9mm}|\hspace{5mm} = \hspace*{5mm}    |\hspace*{6mm} f(n)\hspace*{6mm}|\newline
|\hspace{1,5mm} \p{0} \hspace{3mm}  \p{1} \hspace{3mm} \p{0} \hspace{0,5mm}| \hspace{10,7mm}| \hspace{1,7mm} f(n-\p{2}) \hspace{1mm}|\hspace{16mm} |\hspace*{2,5mm} f(n-\p{1})\hspace*{4,5mm} |
\end{tcolorbox}

\vspace{\baselineskip}
Они рассчитываются так же, как и в типе 2. Если для вас это сложно, попробуйте вычислить их на бумаге.

\vspace{\baselineskip}

\textbf{Тип 4:}

\vspace{\baselineskip}
Жизнь становится адски сложной, и госпожа Проблема теперь просит вас найти f(n) = f(n-\p{1}) + f(n-\p{2}) + c, где \textbf{c} - любая константа.\newline
На этот раз это что-то новое, ведь в прошлом мы видели лишь то, что после умножения каждое состояние в \textbf{А}  переходит в свое следующее состояние в \textbf{B}.

\vspace{\baselineskip}
\begin{tcolorbox}
f(n)\hspace{2,5mm} = \hspace{2,5mm}f(n-\p{1})\hspace*{2,3mm} + \hspace*{0,6mm} f(n-\p{2})\hspace*{1,5mm} + \hspace*{1,5mm}c\newline
f(n+\p{1})\hspace*{3mm}=\hspace*{2mm} f(n)\hspace*{1,9mm} +\hspace*{1,9mm} f(n-\p{1})\hspace*{1,7mm} +\hspace*{1,5mm} c\newline
f(n+\p{2})\hspace*{2mm} = \hspace*{2mm}f(n+\p{1})\hspace*{0,9mm} +\hspace{0,9mm} f(n)\hspace*{1,5mm} +\hspace*{1,5mm} c\newline
. . . . . . . . . . . . . . . . . . . . so on
\end{tcolorbox}

\vspace{\baselineskip}
Итак, мы не сможем продолжить в той же манере, что и раньше, но что будет, если добавить \textbf{c} в качестве состояния:

\vspace{\baselineskip}
\begin{tcolorbox}
  
\hspace{13,4mm} |\hspace{3,5mm}f(n)\hspace{4,4mm}|\hspace{12,5mm}|\hspace{2,5mm}f(n+\p{1})\hspace{3mm}|\newline
\hspace*{3,5mm} M $\times$ |\hspace{1mm} f(n-\p{1})\hspace{1mm} |\hspace{3,1mm} = \hspace*{3,1mm}|\hspace{4mm}f(n)\hspace{7mm}|\newline
\hspace*{14,6mm}|\hspace*{5,3mm} c \hspace*{5,3mm}|\hspace*{10,7mm} |\hspace*{6mm} c \hspace*{6mm} |


\end{tcolorbox}

\vspace{\baselineskip}
Теперь не так уж трудно составить \textbf{M}. Вот как это делается, но не забудьте проверить самостоятельно:

\vspace{\baselineskip}
\begin{tcolorbox}
|\hspace{1,5mm} \p{1} \hspace{3mm}  \p{1} \hspace{3mm} \p{1} \hspace{0,5mm}| \hspace{10,5mm}| \hspace{3mm} f(n) \hspace{3mm}|\hspace{16mm} |\hspace*{2,5mm} f(n+\p{1})\hspace*{2,9mm} |\newline
|\hspace{1,5mm} \p{1}\hspace{5,8mm}\p{0}\hspace{5,6mm}\p{0}\hspace{0,6mm} |\hspace*{3mm} $\times$\hspace{3mm}     |\hspace*{2,5mm} f(n-\p{1})\hspace*{2,9mm}|\hspace{5mm} = \hspace*{5mm}    |\hspace*{6mm} f(n)\hspace*{6mm}|\newline
|\hspace{1,5mm} \p{0} \hspace{3mm}  \p{0} \hspace{3mm} \p{1} \hspace{0,5mm}| \hspace{10,7mm}| \hspace{5,5mm} c \hspace{5,5mm}|\hspace{16mm} |\hspace*{7,1mm} c \hspace*{7,1mm} |
\end{tcolorbox}


\vspace{\baselineskip}
\textbf{Тип 5:}

\vspace{\baselineskip}
Давайте соберем все это: требуется найти f(n) = a $\times$ f(n-\p{1}) + c $\times$ f(n-\p{3}) + d $\times$ f(n-\p{4}) + e. Давайте оставим это в качестве упражнения для вас. Сначала попробуйте найти состояния и матрицу \textbf{M}. И проверить, соответствует ли она вашему решению. Также найдите матрицы \textbf{A} и \textbf{В}.

\vspace{\baselineskip}
\begin{tcolorbox}

|\hspace{1mm} a \hspace{0,5mm} \p{0}\hspace*{1,5mm} c \hspace*{0,5mm} d \hspace*{0,5mm} \p{1} |\newline
|\hspace*{1mm} \p{1} \hspace*{0,5mm} \p{0} \hspace*{1,5mm}\p{0}\hspace*{1,6mm} \p{0}\hspace*{2,2mm} \p{0}  |\newline
|\hspace*{1mm} \p{0} \hspace*{0,5mm} \p{1} \hspace*{1,5mm}\p{0} \hspace*{0,5mm} \p{0}\hspace*{3,3mm}\p{0}  |\newline
|\hspace*{1mm} \p{0}\hspace*{1,7mm} \p{0}\hspace*{1,5mm} \p{1} \hspace*{0,7mm} \p{0} \hspace*{0,6mm} \p{0}  |\newline
|\hspace*{1mm} \p{0}\hspace*{1,7mm} \p{0}\hspace*{1,7mm} \p{0} \hspace*{0,5mm} \p{0} \hspace*{1,9mm}\p{1}  |\newline

\end{tcolorbox}
\vspace{\baselineskip}
\textbf{Тип 6:}

\vspace{\baselineskip}
Иногда рекурсия дана в таком виде:

\vspace{\baselineskip}
\begin{tcolorbox}

f(n) = f(n-\p{1})\hspace{3mm}   -> \textbf{if} n is odd\newline
f(n) = f(n-\p{2})\hspace*{3mm}   -> \textbf{if} n is even
\end{tcolorbox}

\vspace{\baselineskip}
В сокращенной форме:

 \vspace{\baselineskip}
\begin{tcolorbox}
   f(n) = (n\&\p{1}) $\times$ f(n-\p{1}) + (!(n\&\p{1})) $\times$ f(n-\p{2})
\end{tcolorbox}

\vspace{\baselineskip}
Здесь мы можем разделить функции на основе четности (четные и нечетные) и хранить 2 разные матрицы для обеих из них, а затем вычислить их отдельно.

\vspace{\baselineskip}
\textbf{Тип 7:}

\vspace{\baselineskip}
Чувствуете себя слишком уверенно? Повезло вам. Иногда нам может потребоваться обработать более одной рекурсии там, где это необходимо. Например, пусть рекурсия задана следующим образом:

\vspace{\baselineskip}

\begin{tcolorbox}
   g(n) = 2g(n-\p{1}) + 2g(n-\p{2}) + f(n)            
\end{tcolorbox}

\vspace{\baselineskip}
Здесь рекурсия \textit{g(n)} зависит от f(n), и может быть вычислена в той же матрице, но с увеличенными размерами. Давайте сначала построим из этого выражения матрицы \textbf{A} и \textbf{B}.

\vspace{\baselineskip}
\begin{tcolorbox}
\hspace{3,5mm} Matrix A \hspace{20mm}Matrix B \newline
\hspace*{3mm}|\hspace{5mm}  g(n)  \hspace*{5mm}|  \hspace*{13mm} |\hspace*{2mm} g(n+\p{1})\hspace*{2mm} |\newline
\hspace*{3mm}|\hspace*{3mm} g(n-\p{1})\hspace*{3,5mm} |  \hspace*{13mm} |\hspace*{4mm}  g(n) \hspace{4mm}  |\newline
\hspace*{3mm}|\hspace{3mm}  f(n+\p{1})  \hspace*{2,5mm}|  \hspace*{13mm} |\hspace*{2,5mm} f(n+\p{2})\hspace*{2,4mm} |\newline
\hspace*{3mm}|\hspace*{5,3mm} f(n)\hspace*{5,4mm} |  \hspace*{13mm} |\hspace*{2,5mm}  f(n+\p{1}) \hspace{1,1mm}  |
\end{tcolorbox}



\vspace{\baselineskip}
Здесь g(n+\p{1}) = 2g(n-\p{1}) + f(n+\p{1}) и  f(n+\p{2}) = 2f(n+\p{1}) + 2f(n). Теперь, используя процессы, описанные выше, мы можем найти искомую матрицу \textbf{M}, которая будет выглядеть следующим образом:

\vspace{\baselineskip}
\begin{tcolorbox}

|\hspace{0,3mm} \p{2} \hspace{0,3mm} \p{2} \hspace{0,3mm} \p{1} \hspace{0,3mm} \p{0} \hspace*{0,3mm} |\newline
|\hspace*{0,3mm} \p{1} \hspace*{0,3mm} \p{0} \hspace*{0,3mm} \p{0} \hspace*{0,3mm} \p{0} \hspace*{0,3mm} |\newline
|\hspace*{0,3mm} \p{0} \hspace*{0,3mm} \p{0} \hspace*{0,3mm} \p{2} \hspace*{0,3mm} \p{2} \hspace*{0,3mm} |\newline
|\hspace*{0,3mm} \p{0} \hspace*{0,3mm} \p{0} \hspace*{0,3mm} \p{1} \hspace*{0,3mm} \p{0} \hspace*{0,3mm} |
\end{tcolorbox}
\vspace{\baselineskip}
Итак, это основные категории рекуррентных отношений, которые для решения используют эту простую методику.

