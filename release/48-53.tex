%Иванов Ф

\chapter*{Глава 10: Обходы графов}
\section*{Раздел 10.1: Функция обхода поиска в глубину}

\vspace{\baselineskip}
Функция принимает аргумент текущего индекса узла, список смежности (хранящийся в векторе векторов в этом примере), и вектор логического типа, чтобы отслеживать, какой узел был посещен.

\vspace{\baselineskip}
\begin{tcolorbox}
\begin{minted}{C}
// Depth-first search - поиск в глубину
void dfs(int node, vector<vector<int>>* graph, vector<bool>* visited) {
	// проверить, был ли узел посещен ранее
	if((*visited)[node])
		return;

	/* установить как посещенный, чтобы избежать посещения одного и того же 
	узла дважды*/ 
	(*visited)[node] = true;

	// выполнить какое-то действие здесь
	cout << node;

	// перейти к соседним узлам в глубину рекурсивно
	for(int i = 0; i < (*graph)[node].size(); ++i)
		// запускаем рекурсивно для каждого узла
		dfs((*graph)[node][i], graph, visited);
}
\end{minted}
\end{tcolorbox}

\chapter*{Глава 11: Алгоритм Дейкстры}

\section*{Раздел 11.1: Алгоритм Дейкстры по поиску~кратчайшего \ пути}

\vspace{\baselineskip}
\textit{Перед тем как продолжить, рекомендуется иметь краткое представление о матрице смежности и поиске в ширину.}

\vspace{\baselineskip}
\href{https://en.wikipedia.org/wiki/Dijkstra\%27s_algorithm}{\underline{Алгоритм Дейкстры}} известен как алгортим поиска кратчайшего пути из одного узла-источника во все остальные. Он используется для нахождения кратчайшего пути между узлами в графе, который может представлять собой, например, сети дорог. Он был сформулирован

\href{https://en.wikipedia.org/wiki/Edsger_W._Dijkstra}{\underline{Эдсгером В. Дейкстрой}} в 1956 и опубликован три года спустя.

\vspace{\baselineskip}
Мы можем найти кратчайший путь, используя алгоритм поиска в ширину. Этот алгоритм работает хорошо, но проблема заключается в том, что он предполагает, что стоимость прохождения каждого пути одинакова, а это означает, что вес каждого ребра должен быть одинаковым. Алгоритм Дейкстры помогает нам найти кратчайший путь там, где стоимость прохождения каждого пути неодинакова.

\vspace{\baselineskip}
Для начала мы посмотрим, как модифицировать поиск в ширину, чтобы написать алгоритм Дейкстры, затем добавим очередь с приоритетом, чтобы окончательно сделать из него алгоритм Дейкстры.

\vspace{\baselineskip}
Пусть расстояние до каждого узла от вершины-источника храниться в массиве \textbf{d[]}. Так, например, \textbf{d[3]} означает, что необходимо \textbf{d[3]} времени, чтобы достигнуть \textbf{узла 3} из вершины-\textbf{источника}. Если мы не знаем расстояние, мы будем хранить \textit{бесконечность} в \textbf{d[3]}. Также, пусть \textbf{cost[u][v]} обозначает стоимость пути \textbf{u-v}. Это означает, что требуется \textbf{cost[u][v]}, чтобы пройти из узла \textbf{u} в узел \textbf{v}.

\vspace{\baselineskip}
\includeimage{0.3}{images/49.pdf}

\vspace{\baselineskip}
Нам необходимо понять, что такое Релаксация Ребра. Пусть, от вашего дома, который является \textbf{вершиной-источником}, до места \textbf{A} на дорогу требуется 10 минут. И до места \textbf{B} требуется 25 минут. Мы имеем,

\vspace{\baselineskip}
\begin{tcolorbox}
\begin{minted}{C} 
d[A] = 10
d[B] = 25
\end{minted}
\end{tcolorbox}

\vspace{\baselineskip}
Теперь скажем, что необходимо 7 минут, чтобы добраться от места \textbf{A} до места \textbf{B}, это означает:

\vspace{\baselineskip}
\begin{tcolorbox}
\begin{minted}{C} 
cost[A][B] = 7
\end{minted}
\end{tcolorbox}

\vspace{\baselineskip}
Затем мы можем попасть в место \textbf{B} из \textbf{вершины-источника}, перейдя в место \textbf{A} из \textbf{вершины-источника}, a затем из места \textbf{A}, в место \textbf{B}, что займет 10 + 7 = 17 минут, вместо 25 минут. Итак,

\vspace{\baselineskip}
\begin{tcolorbox}
\begin{minted}{C} 
d[A] + cost[A][B] < d[B]
\end{minted}
\end{tcolorbox}

\vspace{\baselineskip}
Тогда мы присваиваем,

\vspace{\baselineskip}
\begin{tcolorbox}
\begin{minted}{C} 
d[B] = d[A] + cost[A][B]
\end{minted}
\end{tcolorbox}

\vspace{\baselineskip}
Это называется релаксацией. Если мы идем из узла \textbf{u} в узел \textbf{v} и\textbf{ d[u] + cost[u][v] < d[v]}, то мы присваиваем \textbf{d[v] = d[u] + cost[u][v]}.

\vspace{\baselineskip}
При поиске в ширину нам не нужно было посещать какой-либо узел дважды. Мы только проверяли, посещен узел или нет. Если он не был посещен, мы помещали узел в очередь, отмечали его как посещенный и инкрементировали расстояние на 1. В алгоритме Дейкстры, мы можем поместить узел в очередь и вместо того, чтобы отметить его как посещенный, мы релаксируем или обновляем новое ребро. Посмотрим на один пример:

\vspace{\baselineskip}
\includeimage{0.9}{images/50_1.pdf}


\vspace{\baselineskip}
Давайте предположим, что \textbf{Узел 1} является \textbf{Источником}. Тогда,

\vspace{\baselineskip}
\begin{tcolorbox}
\begin{minted}{C} 
d[1] = 0
d[2] = d[3] = d[4] = infinity (or a large value)
\end{minted}
\end{tcolorbox}

\vspace{\baselineskip}
Мы присваиваем \textbf{d[2], d[3]} и\textbf{ d[4]} значение бесконечности, потому что мы еще не знаем расстояние. И расстояние от \textbf{вершины-источника} до нее же самой, конечно же, равно 0. Теперь мы идем в другие узлы из \textbf{вершины-источника} и, если мы можем присвоить им новое значение, то мы помещаем их в очередь. Например, мы проходим \textbf{ребро 1-2}. Так как \textbf{d[1] + 2 < d[2]}, мы имеем \textbf{d[2] = 2}. Схожим образом мы пройдем через \textbf{ребро 1-3}, которое делает \textbf{d[3] = 5}.

\vspace{\baselineskip}
Мы четко видим, что 5 не является кратчайшим расстоянием, которое мы можем пересечь, чтобы попасть в \textbf{узел 3}. Поэтому обход узла только один раз, как в поиске в ширину, здесь не работает. Если мы перейдем от \textbf{узла 2} к \textbf{узлу 3}, используя \textbf{ребро 2-3}, мы можем присвоить значение \textbf{d[3] = d[2] + 1 = 3}. Таким образом, мы можем заметить, что один узел может быть обновлен много раз. Сколько раз вы спрашиваете? Максимальное число обновлений узла равно полустепени захода узла.

\vspace{\baselineskip}
Давайте посмотрим псевдокод для посещения любого узла несколько раз. Мы просто модифицируем поиск в ширину:

\vspace{\baselineskip}
\begin{tcolorbox}
\begin{minted}{C} 
// модифицированный поиск в ширину
procedure BFSmodified(G, source):
	// объект очереди
	Q = queue()
	// массив дистанций между source и другими узлами
	distance[] = infinity
	// добавление текущей вершины
	Q.enqueue(source)
	// инициализация дистанции для выбранного узла 
	distance[source] = 0
	// пока очередь не пустая 
	while Q is not empty
		// выбираем первый элемент из очереди
		u <- Q.pop()
		for all edges from u to v in G.adjacentEdges(v) do
			/* если текущая стоимость прохода до проверяемого узла 
			меньше зафиксированной, то обновляем зафиксированное значение */
			if distance[u] + cost[u][v] < distance[v]
				distance[v] = distance[u] + cost[u][v]
			end if
		end for
	end while
return distance
\end{minted}
\end{tcolorbox}

\vspace{\baselineskip}
Этот код может быть использован для нахождения кратчайшего пути до всех узлов из вершины-источника. Сложность этого кода не так хороша. Вот почему,

\vspace{\baselineskip}
При поиске в ширину, когда мы идем из \textbf{узла 1} во все другие узлы, мы следуем методу первым пришел, первым обслужен. Например, мы пошли в \textbf{узел 3} из вершины-источника до обработки \textbf{узла 2}. Если мы идем в \textbf{узел 3} из \textbf{вершины-источника}, мы присваиваем значение \textbf{узлу 4}, соответствующее \textbf{5 + 3 = 8}. Когда мы снова обновляем значение\textbf{ узла 3} из \textbf{узла 2}, нам нужно снова присвоить значение \textbf{узлу 4}, соответствующее \textbf{3 + 3 = 6}! Итак, \textbf{узел 4} обновлен дважды.

\vspace{\baselineskip}
\textit{Дейкстра }предложил вместо метода \textit{первым пришел, первым обслужен}, сначала обновлять ближайшие узлы, тогда потребуется меньше обновлений. Если мы обработали \textbf{узел 2} до этого, тогда\textbf{ узел 3} должен был быть обновлен до и после обновления \textbf{узла 4} соответственно, тогда мы бы с легкостью получили кратчайшее расстояние! Идея состоит в том, чтобы выбирать из очереди узел, который является ближайшим к\textbf{ вершине-источнику}. Итак мы будем тут использовать (очередь с приоритетом), чтобы при удалении элемента из очереди, она выдавала нам ближайший узел u к \textbf{вершине-источнику}. Как она сделает это? Она будет проверять значение \textbf{d[u]} внутри.

\vspace{\baselineskip}
Давайте посмотрим на псевдокод:

\vspace{\baselineskip}
\begin{tcolorbox}
\begin{minted}{C} 
procedure dijkstra(G, source):
	// объект очереди с приоритетами
	Q = priority_queue()
	// массив дистанций между source и другими узлами
	distance[] = infinity
	// добавление текущей вершины

	Q.enqueue(source)
	// инициализация дистанции для выбранного узла 
	distance[source] = 0
	// пока очередь не пустая 
	while Q is not empty
		// заносим ближайшую вершину к вершине-источнику в u 
		u <- nodes in Q with minimum distance[]
		// удаляем ближайшую вершину из очереди
		remove u from the Q
		for all edges from u to v in G.adjacentEdges(v) do
			/* если текущая стоимость прохода до проверяемого узла меньше
			зафиксированной, то обновляем зафиксированное значение */
			if distance[u] + cost[u][v] < distance[v]
				distance[v] = distance[u] + cost[u][v]
				// заносим примыкающий узел
				Q.enqueue(v)
			end if
		end for
	end while
Return distance
\end{minted}
\end{tcolorbox}

\vspace{\baselineskip}
Псевдокод возвращает расстояние до всех других узлов от\textbf{ вершины-источника}. Если мы хотим узнать расстояние до одного узла \textbf{v}, мы можем просто вернуть значение, когда \textbf{v} удаляется из очереди.

\vspace{\baselineskip}
А теперь узнаем, работает ли алгоритм Дейкстры при наличии отрицательного ребра? Если в графе присутствует отрицательный цикл, то получим бесконечный цикл, так как отрицательный цикл будет продолжать уменьшать значение каждый раз. Даже если в графе присутствует лишь отрицательное ребро, алгоритм Дейкстры не будет работать, за исключением, если мы вернемся после того, как цель установлена. Но тогда это не будет алгоритмом Дейкстры. Нам понадобится алгоритм Беллмана-Форда для обработки отрицательных ребер/циклов.

\vspace{\baselineskip}
\textbf{Сложность:}

\vspace{\baselineskip}
Сложность поиска в ширину равна\textbf{ O(log(V+E))}, где \textbf{V} - это число узлов и \textbf{E} --- это число ребер. Для алгоритма Дейкстры сложность схожая, но сортировка очереди с приоритетом требует \textbf{O(logV)}. Поэтому общая сложность: \textbf{O(Vlog(V)+E)}.

\vspace{\baselineskip}
Ниже расположен Java пример, использующий матрицу смежности,для реализации алгоритма Дейкстры по поиску кратчайшего пути:

\vspace{\baselineskip}
\begin{tcolorbox}
\begin{minted}[obeytabs=true,tabsize=3]{Java} 

import java.util.*;
import java.lang.*;
import java.io.*;

class ShortestPath
{
	// количество вершин  <-> размерность матрицы
	static final int V=9;

	int minDistance(int dist[], Boolean sptSet[])
	{
		/* ищем минимальное расстояние до непомеченных вершин */
		// ставим планку для сравниваемого значения и индекса
		int min = Integer.MAX_VALUE, min_index=-1;
		
		// проход по строкам матрицы
		for (int v = 0; v < V; v++)
			// если ребро не учтено и минимально на текущий момент
			if (sptSet[v] == false && dist[v] <= min)
			{
				// устанавливаем новую планку
				min = dist[v];
				min_index = v;
			}
		// возвращаем индекс самого малого веса ребра
		return min_index;
	}


	void printSolution(int dist[], int n)
	{
		/* вывод результата в виде списка */
		System.out.println("Vertex Distance from Source");

		// первый столбец - индекс вершины; второй - минимальной дистанции
		for (int i = 0; i < V; i++)
			System.out.println(i+" \t\t "+dist[i]);
	}


	void dijkstra(int graph[][], int src)
	{
		// инициализация булева массива по количеству вершин
		Boolean sptSet[] = new Boolean[V];

			for (int i = 0; i < V; i++)
			{
				// присваиваем каждому ребру максимальный вес и ставим флаг
				dist[i] = Integer.MAX_VALUE;
				// “непосещение” текущего ребра
				sptSet[i] = false;
			}

		// стартовое состояние одной из вершин
		dist[src] = 0;

		for (int count = 0; count < V-1; count++)
		{
			int u = minDistance(dist, sptSet);
	
			// помечаем вершину с минимальным расстоянием
			sptSet[u] = true;

			/* ищем меньшие варианты обхода,
			обновляем ранее установленное расстояние */
			for (int v = 0; v < V; v++)
				if (!sptSet[v] && graph[u][v]!=0 &&
						dist[u] != Integer.MAX_VALUE &&
						dist[u]+graph[u][v] < dist[v])
					dist[v] = dist[u] + graph[u][v];
		}
		// выводим результат
		printSolution(dist, V);
	}


	public static void main (String[] args)
	{
		// инициализируем двумерный массив симметричной матрицы смежности
		int graph[][] = new int[][]{{0, 4, 0, 0, 0, 0, 0, 8, 0},
																{4, 0, 8, 0, 0, 0, 0, 11, 0},
																{0, 8, 0, 7, 0, 4, 0, 0, 2},
																{0, 0, 7, 0, 9, 14, 0, 0, 0},
																{0, 0, 0, 9, 0, 10, 0, 0, 0},
																{0, 0, 4, 14, 10, 0, 2, 0, 0},
																{0, 0, 0, 0, 0, 2, 0, 1, 6},
																{8, 11, 0, 0, 0, 0, 1, 0, 7},
																{0, 0, 2, 0, 0, 0, 6, 7, 0}
	};

	ShortestPath t = new ShortestPath();

	t.dijkstra(graph, 0);
	}
}
\end{minted}
\end{tcolorbox}

\vspace{\baselineskip}
Ожидаемый вывод программы:

\vspace{\baselineskip}
\begin{tcolorbox}
\begin{minted}{C} 

Vertex	 Distance from Source
	0				0
	1				4
	2				12
	3				19
	4				21
	5				11
	6				9
	7				8
	8				14

\end{minted}
\end{tcolorbox}


