% Ивенкова
\chapter*{Глава 4: Деревья}

\vspace{\baselineskip}
\section*{Раздел 4.1: Типичное представление анарного дерева}

\vspace{\baselineskip}
Обычно мы представляем анарное дерево (один узел с потенциально неограниченным числом дочерних элементов) в виде двоичного дерева (один узел ровно с парой детей). «Следующий» ребенок считается родным братом. Обратите внимание, что если дерево является двоичным, то такое представление создает дополнительные узлы.

\vspace{\baselineskip}
Затем мы повторяем итерацию над братьями и сестрами и возвращаемся к детям. Поскольку большинство деревьев относительно мелкие, имеем много детей только на нескольких уровнях иерархии, что приводит к оптимизации кода. Обратите внимание, что человеческие родословные являются исключением (много уровней предков и только несколько детей на уровень).

\vspace{\baselineskip}
При необходимости можно сохранить обратные указатели, чтобы по дереву можно было подняться наверх. Такие деревья сложнее хранить.

\vspace{\baselineskip}
Обратите внимание, что нормально иметь одну функцию для вызова у корня и рекурсивную функцию с дополнительными параметрами, в данном случае глубиной дерева.

\vspace{\baselineskip}
\begin{tcolorbox}
\begin{minted}{C}
struct node // структура узла дерева
{
	struct node *next; // указатель на структуру типа node, 
					   // являющейся братом для данного узла
	struct node *child;// указатель на структуру типа node, 
					   // являющейся ребенком для данного узла
	std::string data; // строка данных
}

void printtree_r(struct node *node, int depth) // функция вывода дерева, 
// которая принимает параметры указатель на узел и уровень отступа
{
	int i;
	while(node) // цикл действующий пока узел не пустой
	{
		if(node->child) // если ребенок не пустой
		{
			for(i=0;i<depth*3;i++) // отступ длиной в уровень узла, 
			// умноженный на 3
				printf(" ");
			printf("{\n");
			printtree_r(node->child,depth+1); // рекурсивный вызов 
			// функции вывода дерева с ребенком в качестве аргумента
			for(i=0;i<depth;depth+1);  // отступ длиной на один 
			// уровень больше
				printf(" ");
			printf("{\n");
		}
		for(i=0;i<depth*3;i++) // отступ длиной в уровень узла
			printf(" ");
		printf("%s\n", node->data.c_str());// вывод значения узла

		node = node->next; // переход к брату
		
	}
}

void printtree(node *root) // Вывод дерева начиная с корня			
{
	printtree_r(root, 0); // Вызов функции вывода с произвольной 
	// позиции с корнем, как узлом и нулевым начальным отступом
}
\end{minted}
\end{tcolorbox}

\vspace{\baselineskip}
\section*{Раздел 4.2: Введение}

\vspace{\baselineskip}
\href{https://vk.cc/atDAwi}{\underline{Деревья}} являются подтипом более общей структуры данных узловых графов.


%\newpage
\includeimage{0.7}{images/20.pdf}

\newpage
Чтобы быть деревом, граф должен удовлетворять двум требованиям:

\begin{itemize}
  \item {\bfseries Он является ациклическим.} Он не содержит циклов (или "зацикливания").
  \item {\bfseries Он подключен.} Для любой заданной вершины на графе доступна каждая вершина. Все узлы доступны через один путь в графе.
\end{itemize}

\vspace{\baselineskip}
Древовидная структура данных довольно распространена в компьютерной науке. Деревья используются для моделирования многих различных алгоритмических структур данных, таких как обычные бинарные деревья, красно-черные деревья, B-деревья, AB-деревья,куча, 23-деревья и многие другие.

\vspace{\baselineskip}
Принято называть дерево "корневым":

\vspace{\baselineskip}
\begin{tcolorbox}
Выбирается {\color{Purple}{один}} узел, который будет называться "корнем". 

Нарисовав "корень" на вершине, 

создаем нижние слои для каждой ячейки на графе в зависимости от их расстояния до корня - чем больше расстояние, тем ниже листья (пример выше)
\end{tcolorbox}


\vspace{\baselineskip}
Общий символ для деревьев: Т

\section*{Раздел 4.3: Проверка, одинаковы два двоичных дерева или нет}

\vspace{\baselineskip}
Например, если входные данные такие:

\vspace{\baselineskip}
{\bfseries Пример: 1}

\vspace{\baselineskip}
a) \ \ \ \ \ \ \ \ \ \ \ \ \ \ \ \ \ \ \ \ \ \ \ \ \ \ \ \ \ \ \ \ \ \ \ \ \ \ \ \ \ \ \ \ \ \ \ \ \ \ \ \ \ \ \ \ \ \ \ \ \ \ b)
 
\includeimage{1}{images/21_1.pdf} \ \ \ \ \ \includeimage{1}{images/21_1.pdf}

{\bfseries Результатом должна быть истина.}

\newpage
{\bfseries Пример: 2}

\vspace{\baselineskip}
Если входные данные такие:

\vspace{\baselineskip}
a) \ \ \ \ \ \ \ \ \ \ \ \ \ \ \ \ \ \ \ \ \ \ \ \ \ \ \ \ \ \ \ \ \ \ \ \ \ \ \ \ \ \ \ \ \ \ \ \  b)

\includeimage{1}{images/21_2.pdf} \ \ \ \ \ \includeimage{1}{images/21_3.pdf}

{\bfseries Результатом должна быть ложь.}

\vspace{\baselineskip}
Псевдокод для этого случая:

\vspace{\baselineskip}
\begin{tcolorbox}
\begin{minted}{C}
bookean sameTree(node root1, node root2){ // функция сравнивающая два дерева,
								// принимает в себя указатели на узлы деревьев	
if(root1 == NULL && root2 == NULL) // проверяет пустые ли два узла	
	return true; // если оба узла пустые, то эти 2 поддерева равны 
			 // и мы возвращаем истину		
if(root1 == NULL || root2 == NULL) // если только один из узлов пустой		
	return false; //то поддеревья не равны и мы возвращаем ложь
// если оба поддерева не пустые то сравниваем их значение, и равны ли левое и
// правое поддверево			
if(root1->data == root2->data 
	&& sameTree(root1->left,root2->left) //рекурсивно сравниваем поддеревья
	&& sameTree(root1->right, root2->right))	
	return true;
}
\end{minted}
\end{tcolorbox}

\chapter*{Глава 5: Поиск в двоичных деревьях}

Двоичное дерево - это дерево, в котором каждый узел имеет максимум два дочерних элемента. Двоичное дерево поиска (BST) - это двоичное дерево, элементы которого расположены в специальном порядке. В каждом BST все значения (т.е. ключи) в левом поддереве меньше по значению, чем в правом

\vspace{\baselineskip}
\section*{Раздел 5.1: Дерево поиска двоичных файлов - Вставка (Python)}

\vspace{\baselineskip}
Это простая реализация вставки Binary Search Tree  с использованием Python.

\vspace{\baselineskip}
Пример: \href{http://i.stack.imgur.com/3NG0e.gif}{\underline{анимация}}

\vspace{\baselineskip}
Следуя за фрагментом кода, каждое изображение показывает визуализацию выполнения кода, что облегчает представление того, как это происходит.

\vspace{\baselineskip}
\begin{tcolorbox}
\begin{minted}{Python}
class Node:		# класс узла поддерева
	def __init__(self, val): # Поля класса
		self.l_child = None # указатель на узел левого поддерева
		self.r_child = None # указатель на узел правого поддерева
		self.data = val # значение 

\end{minted}
\end{tcolorbox}

\vspace{\baselineskip}
\includeimage{0.8}{images/23.pdf}

\vspace{\baselineskip}
\begin{tcolorbox}
\begin{minted}{Python}
def insert(root, node): # добавление элемента в дерево 
						# принимает в себя корень и новый узел
	if root is None: # пустой ли корень
		root = node # если корень пуст, то присваиваем ему введенный узел
	else:
		if root.data > node.data: # больше ли элемент данного узла
			if root.l_child is None: # пустой ли левый ребенок
				root.l_childe = node # присваиваем ему узел
			else: #если левый узел не пустой
				insert(root.l_child, node) # повторяем функцию 
										  # для левого поддерева
		else:
			if root.r_child is None: # пустой ли правый ребенок
				root.r_childe = node # присваиваем ему узел
			else: #если  правый узел не пустой
				insert(root.r_child, node) # повторяем функцию 
										  # для правого поддерева
\end{minted}
\end{tcolorbox}

\vspace{\baselineskip}
\includeimage{0.8}{images/24_1.pdf}

\vspace{\baselineskip}
\begin{tcolorbox}
\begin{minted}{Python}
def in_order_print(root): # инфиксный вывод дерева
	if not root: # пустой ли корень
		return
	in_order_print(root.l_child) # рекурсивно вызываем функцию 
								# с левым ребенком
print root.data # выводим данные
in_order_print(root.r_child) # рекурсивно вызываем функцию с правым ребенком
\end{minted}
\end{tcolorbox}

\includeimage{0.8}{images/24_2.pdf}

\vspace{\baselineskip}
\begin{tcolorbox}
\begin{minted}{Python}
def pre_order_print(root): # префиксный вывод дерева
	if not root: # пустой ли корень
		return
	print root.data # выводим данные
	pre_order_print(root.l_child) # рекурсивно вызываем функцию 
								 # с левым ребенком
	pre_order_print(root.r_child) # рекурсивно вызываем функцию 
								 # с правым ребенком

\end{minted}
\end{tcolorbox}

%\newpage
\vspace{\baselineskip}
\includeimage{0.8}{images/25_1.pdf}

\vspace{\baselineskip}
\section*{Раздел 5.2: Дерево двоичного поиска - удаление (C++)}

\vspace{\baselineskip}
Перед началом удаления, я хочу рассказать, что является деревом двоичного поиска (BST). Каждый узел в BST может иметь максимум две вершины (левая и правая дочерние). Левое поддерево вершины имеет ключ, меньший или равный ключу родительского узла. Правое поддерево узла имеет ключ больше, чем ключ родительского узла.

\vspace{\baselineskip}
Удаление узла в дереве при сохранении {\bfseries свойства} самого {\bfseries дерева двоичного поиска.}

\vspace{\baselineskip}
\includeimage{0.2}{images/25_2.pdf}

\vspace{\baselineskip}
При удалении узла необходимо учитывать три случая:

\vspace{\baselineskip}
\begin{itemize}
  \item Случай 1: Удаляемый узел - это узел листа (узел со значением 22).
  \item Случай 2: У удаляемого узла есть один дочерний элемент (узел со значением 26). 
  \item Случай 3: У удаляемого узла есть оба дочерних узла (узел со значением 49).
\end{itemize}

\vspace{\baselineskip}
{\bfseries Пояснения по случаям:}

\begin{enumerate} 
	\item Когда удаляемый узел является узлом листа, просто удалите узел и передайте nullptr его родительскому узлу.
	\item Когда удаляемый узел имеет только один дочерний, скопируйте дочернее значение в значение узла и удалите ребёнка {\bfseries (преобразовано в случай 1)}
	\item Если удаляемый узел имеет двух детей, то минимальное количество детей из его правого поддерева может быть скопировано в узел, а затем минимальное значение может быть удалено из правого поддерева узла {\bfseries (преобразовано в случай 2)}
\end{enumerate}

\vspace{\baselineskip}
{\bfseries Примечание:} минимум в правом поддереве может иметь максимум одного ребенка, и этот правый ребенок, если у него есть левый ребенок, либо не минимальное значение, либо не следует свойству BST.

\vspace{\baselineskip}
Структура узла в дереве и код для удаления:

\vspace{\baselineskip}
\begin{tcolorbox}
\begin{minted}{C}
struct node // структура узла бинарного дерева
{
	int data; // данные
	node *left, *right; // указатели на левое и правое поддерево
};
node* delete_node(node *root, int data) // удаление данных из бинарного дерева
{
	if(root == nullptr) return root; // если девево пустое, то ничего не 
	// делаем
// если данные меньше значения узла, повторяем удаление для левого поддерева
	else if(data < root->data) root->left = delete_node(root->left, data); 
// если данные больше значения узла, повторяем удаление для правого поддерева
	else if(data > root->data) root->right = delete_node(root->right, data);
// если они равны
	else
	{
		if(root->left == nullptr && root->right == nullptr) //1 случай
		{
			free(root);
			root = nullptr;
		}
		else if(root->left == nullptr)					  //2 случай
		{
			node* temp = root;
			root = root->right;
			free(temp);
		}
		else if(root->right ==nullptr)					  //2 случай
		{
			node* temp = root;
			root = root->left;
			free(temp);
		}
		else												//3 случай
		{
			node* temp = root->right;
			while(temp->left != nullptr) temp = temp->left;
			root->data = temp->data;
			root->right = delete_node(root->right, temp->data);
		}
	}
	return root;
}

\end{minted}
\end{tcolorbox}

\vspace{\baselineskip}
Временная сложность приведенного кода - O(h), где h - это высота дерева.

\vspace{\baselineskip}
\section*{Раздел 5.3: Самый низкий предок в BST}

\vspace{\baselineskip}
Рассмотрим BST:

\includeimage{0.2}{images/25_2.pdf}

\vspace{\baselineskip}
Самый низкий родоначальник 22 и 26 - 24

\vspace{\baselineskip}
Самый низкий родоначальник 26 и 49 - 46.

\vspace{\baselineskip}
Самый низкий родоначальник 22 и 24 - 24

\vspace{\baselineskip}
Свойство поиска двоичного дерева может быть использовано для нахождения узлов самого низкого предка

\vspace{\baselineskip}
{\bfseries Псевдокод:}

\vspace{\baselineskip}
\begin{tcolorbox}
\begin{minted}{C}
lowestCommonAncestor(root,node1,node2) //нахождение общего предка для 2 узлов
{
if(root == NULL) return NULL;
else if(node1->data == root->data || node2->data == root->data) //один из узлов
																//корень
	return root; //возвращаем корень
else if((node1->data <= root->data && node->data > root->data) //узлы по разные
	|| (node2->data <= root->data && node1->data > root->data)){ //стороны 
																//от корня
	return root; //возвращаем корень
	}
else if(root->data >max(node1->data,node2->data)){ //узлы меньше корня
//ищем общего предка для этих узлов в левом поддереве
	return lowestCommonAncestor(root->left, node1, node2); 
	}
else{ //узлы больше корня
//ищем общего предка для этих узлов в левом поддереве
	return lowestCommonAncestor(root->right, node1, node2);		
	}
}
\end{minted}
\end{tcolorbox}

\vspace{\baselineskip}
\section*{Раздел 5.4: Поиск в двоичном  дереве - Python}

\vspace{\baselineskip}
\begin{tcolorbox}
\begin{minted}[obeytabs=true,tabsize=3]{Python}
class Node(object): # узел двоичного дерева
	def __init__(self, val):
		self.l_child = None # Указатель на левое поддерево
		self.r_child = None # Указатель на правое поддерево
		self.val = val # Значение

class BinarySearchTree(object): # Функция поиска в бинарном дереве
	def insert(self, root, node):
		if root is None: # если корень пустой, то корень равен узлу
			return node
		if root.val < node.val # если значение узла больше узла корня
			root.r_child = self.insert(root.r_child, node) # ищем в правом поддереве
		else:	# если меньше
			root.l_child = self.insert(root.l_child, node) # ищем в левом поддереве
		return root 

	def in_order_place(self, root): # вывод в инфиксной форме
		if not root: # пустой ли корень
			return None
		else:
			self.in_order_place(root.l_child) # рекурсивно вызываем функцию 
											  # с левым ребенком
			print root.val # выводим данные
			self.in_order_place(root.r_child) # рекурсивно вызываем функцию 
											  # с правым ребенком

	def pre_order_place(self, root):	# вывод в префиксной форме
		if not root: # пустой ли корень
			return None
		else:
			print root.val # выводим данные
			self.pre_order_place(root.l_child) # рекурсивно вызываем функцию 
											   # с левым ребенком
			self.pre_order_place(root.r_child) # рекурсивно вызываем функцию 
											   # с правым ребенком


def post_order_place(self, root):	# вывод в постфиксной форме
		if not root: # пустой ли корень
			return None
		else:
			print root.val # выводим данные
			self.pre_order_place(root.l_child) # рекурсивно вызываем функцию 
											   # с левым ребенком
			self.pre_order_place(root.r_child) # рекурсивно вызываем функцию 
											   # с правым ребенком

\end{minted}
\end{tcolorbox}

\vspace{\baselineskip}
"""Создать другой узел и вставить в него данные"""

\vspace{\baselineskip}
\begin{tcolorbox}
\begin{minted}{Python}
r = Node(3)				# создания корня со значением 3
node = BinarySearchTree()			# создание бинарного дерева
nodeList = [1, 8, 5, 12, 14, 6, 15, 7, 16, 8] # массив вводимых значений
for nd in nodeList:				# заполнение дерева
	node.insert(r, Node(nd))
print "------In order ---------"			# вывод дерева 
print (node.in_order_place(r))
print "------Pre order ---------"
print (node.pre_order_place(r))
print "------Post order ---------"
print (node.post_order_place(r))
\end{minted}
\end{tcolorbox}

\chapter*{Глава 6: Проверка, является дерево BST или нет}

\vspace{\baselineskip}
\section*{Раздел 6.1: Алгоритм проверки, является ли данное двоичное дерево BST}

\vspace{\baselineskip}
Двоичное дерево является BST, если оно удовлетворяет одному из следующих условий:

\begin{enumerate} 
	\item Оно пусто.
	\item У него нет поддерева
	\item В дереве для каждого узла x все ключи (если таковые имеются) в левом поддереве должны быть меньше, чем ключ (x), а все ключи (любые) в правом поддереве должны быть больше, чем ключ(x).
\end{enumerate}

Так что прямолинейный рекурсивный алгоритм был бы такой:

\vspace{\baselineskip}
\begin{tcolorbox}
\begin{minted}[obeytabs=true,tabsize=3]{Python}
is_BST(root): 	# является ли дерево бинарным поиска
	if root == NULL: 	# если дерево пустое то оно BST
		return true
	# Проверка значения левого поддерева
	if root->left != NULL;
		max_key_in_left = find_max_key(root->left)  # максимальное значение 
													# в левом поддереве
		if max_key_in_left > root->key: # если оно больше корня, то оно не BST
			return false

	# Проверка правого поддерева
	if root->right != NULL:
		min_key_in_right = find_min_key(root->right)   # максимальное значение 
													   # в правом поддереве
		if min_key_in_right < root->key: # если оно меньше корня, то оно не BST
			return false
	return is_BST(root->left) && is_BST(root->right) # возвращаем значение 
												# для левого и правого поддерева
\end{minted}
\end{tcolorbox}

\vspace{\baselineskip}
Вышеуказанный рекурсивный алгоритм корректен, но неэффективен, т.к. он обходит каждый узел несколько раз.

\vspace{\baselineskip}
Другой подход к минимизации многократных посещений каждого узла заключается в запоминании минимально и максимально возможных значений ключей в поддереве, которые мы посещаем. Пусть минимально возможное значение любого ключа будет K\_MIN, а максимальное значение будет K\_MAX. Когда мы начинаем с корня дерева, диапазон значений в дереве будет  [K\_MIN, K\_MAX]. Пусть ключ от корня узла будет х, тогда диапазон значений в левом поддереве равен [K\_MIN, x), а диапазон значений в правом поддереве равен (x, K\_MAX]. Мы будем использовать эту идею для разработки более эффективного алгоритма.

\vspace{\baselineskip}
\begin{tcolorbox}
\begin{minted}{Python}
is_BST(root, min, max):  # является ли дерево бинарным в заданном диапазоне
if root == NULL	# если пустое, то является
	return true

# узел выходит за диапазон?
if root->key < min || root->key >max # если нет, то возвращаем ложь
	return false
# проверка, являются ли левое и правое поддерево BST на диапозоне
return if_BST(root->left,min,root->key-1) && is_BST(root->right,root->key+1,max)

\end{minted}
\end{tcolorbox}

\vspace{\baselineskip}
Первоначально он будет называться так:

\vspace{\baselineskip}
\begin{tcolorbox}
\begin{minted}{Python}
is_BST(my_tree_root, KEY_MIN, KEY_MAX)
\end{minted}
\end{tcolorbox}

\vspace{\baselineskip}
Другой подход будет заключаться в обходе двоичного дерева. Если обход по порядку производит отсортированную последовательность ключей, то данное дерево является BST. Чтобы проверить, отсортирована ли последовательность ключей, запомните значение ранее посещенного узла и сравните его с текущим узлом.