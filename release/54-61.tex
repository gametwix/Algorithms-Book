%Мохляков
\chapter*{Глава 12: Поиск пути А*}
\section*{Раздел 12.1: Введение в А*}

А*(А star) - это поисковый алгоритм, который используется для поиска пути от одного узла к другому. Его можно сравнить с такими алгоритмами, как поиск в ширину, алгоритм Дейкстры, поиск в глубину или поиск по первому наилучшему совпадению. Алгоритм А* широко используется в поиске графов из-за более высокой эффективности и точности там, где предварительная обработка данных графа неуместна.

\vspace{\baselineskip}

А* - разновидность поиска по первому наилучшему совпадению (Best-first
search), в котором функция оценки f определяется особым образом.

\vspace{\baselineskip}

$f(n)=g(n)+h(n)$ имеет минимальную стоимость с момента, когда исходный узел, предназначенный для необходимых целей, вынужден проходить через узел n.

\vspace{\baselineskip}

$g(n)$ имеет минимальную стоимость от корневого узла до n.

\vspace{\baselineskip}

$h(n)$ имеет минимальную стоимость от n до ближайшей к n цели.

\vspace{\baselineskip}

А* является информированным методом поиска, и он всегда гарантированно найдет кратчайший путь (путь с минимальной стоимостью) за наименьшее возможное время (если использует \href{https://vk.cc/atzPw1}{\underline{приемлемую эвристику}}). Поэтому он и оптимален, и совершенен. Следующая \href{https://i.stack.imgur.com/TGfc9.gif}{\underline{анимация}} наглядно демонстрирует работу А*.

\vspace{\baselineskip}

\section*{Раздел 12.2: Поиск пути А* в лабиринте без препятствий}

Скажем, что у нас есть сетка размера 4 на 4:
\newpage

\includeimage{0.85}{images/55_1.pdf}

Представим, что перед нами лабиринт. Однако в нем отсутствуют препятствия/стенки. У нас имеется только стартовая точка (зеленая клетка) и конечная точка (красная клетка). Также примем тот факт, что при прохождении нашего пути мы не можем двигаться по диагонали. Так, начиная движение с зеленой клетки, посмотрим, в какие клетки мы можем перейти из неё и пометим эти клетки синим цветом:

\newpage

\includeimage{1}{images/56_1.pdf}

Чтобы выбрать, на какую же клетку нам стоит перейти, мы должны принять во внимание 2 эвристики:
\begin{enumerate} 
 \item Значение “g” . Показатель того, насколько далеко текущий узел находится от зеленой клетки.
 \item Значение “h” . Показатель того, насколько далеко текущий узел находится от красной клетки.
 \item Значение “f” -.Сумма значений “g” и “h”. Это результирующее число, которое и говорит нам, в какой узел (клетку) стоит идти.
\end{enumerate}
Для вычисления этих эвристик мы будем использовать следующую формулу: 

\begin{tcolorbox}
distance = abs(from.x - to.x) + abs(from.y - to.y) 
\end{tcolorbox}
\vspace{\baselineskip}

Эта формула также известна как “\href{https://vk.cc/44BmEO}{\underline{Расстояние городских кварталов}}”.

\vspace{\baselineskip}

Посчитаем значение g клетки, находящейся слева от зеленой клетки:

\begin{tcolorbox}
abs(\p{3} - \p{2}) + abs(\p{2} - \p{2}) = \p{1}
\end{tcolorbox}
\vspace{\baselineskip}

Отлично! Мы получили значение: “1”. А теперь, попробуем посчитать значение “h”:

\begin{tcolorbox}
abs(\p{2} - \p{0}) + abs(\p{2} - \p{0})=\p{4}
\end{tcolorbox}
\vspace{\baselineskip}

Превосходно. А теперь получим значение f: \begin{tcolorbox} \p{1} + \p{4} = \p{5} \end{tcolorbox}

\vspace{\baselineskip}

В итоге мы получили финальное значение: “5”.

\vspace{\baselineskip}

Проделаем аналогичные действия с другими синими клетками. Большое число посередине клетки - значение f, тем временем как числа в верхнем левом и верхнем правом углах - значения g и h соответственно:

%\newpage

\includeimage{1}{images/57_1.pdf}

Мы посчитали значения g , h и f для всех синих клеток(узлов). Что же мы должны выбрать сейчас?

\vspace{\baselineskip}

Мы должны выбрать узел с наименьшим значением f.

\vspace{\baselineskip}

Однако в таком случае мы имеем 2 узла с одинаковыми значениями f, равными 5. Какой из них нам выбрать?

\vspace{\baselineskip}

Один из них мы просто берем случайно, или выставляем список приоритетов. Я обычно предпочитаю расставлять приоритет следующим образом: “Вправо \&gt; Вверх \&gt; Вниз \&gt; Влево”

\vspace{\baselineskip}

Один из узлов со значением 5 поведем нас в направлении “Вниз”, а другой поведет в направлении “Влево”. Так как приоритет направления “Вниз” выше приоритета направления “Влево”, мы выберем тот узел, что повел нас вниз.

\vspace{\baselineskip}

Теперь я помечаю узлы, для которых мы вычислили эвристику, но не перешли в них - оранжевым. Узел, который мы выбрали, я отмечу бирюзовым:

\newpage

\includeimage{1}{images/58_1.pdf}

Хорошо, а теперь вычислим те же эвристики для узлов, расположенных вокруг бирюзовой клетки:

\includeimage{1}{images/58_2.pdf}

И снова, мы выбираем узел, идущий вниз от бирюзовой клетки, так как мы вновь имеем две клетки с одинаковыми f значениями:

\newpage

\includeimage{1}{images/59_1.pdf}

Посчитаем эвристики для единственного соседа бирюзовой клетки:

\includeimage{1}{images/59_2.pdf}

Хорошо, пока мы продолжаем в том же духе, мы будем видеть следующее:

\newpage

\includeimage{1}{images/60_1.pdf}

И еще раз, посчитаем эвристики для соседнего узла:

\includeimage{1}{images/60_2.pdf}

Двинемся сюда:

\newpage