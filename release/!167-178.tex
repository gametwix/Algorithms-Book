% Иванов Ф

\chapter*{Глава 34. Сортировка подсчетом}
\section*{Раздел 34.1: Основная информация о сортировке подсчетом} 

\vspace{\baselineskip}
Сортировка подсчетом — это алгоритм целочисленной сортировки для набора
объектов, который сортируется в соответствии с ключами объектов.

\vspace{\baselineskip}
\textbf{Этапы}

\vspace{\baselineskip}

\begin{enumerate}
\item Строим рабочий массив C, размер которого равен диапазону входного массива A.
\item Перебираем массив A и присваиваем числу C[x] количество вхождений числа x в массив A.
\item Преобразовываем C в массив, где C[x] - количество значений $\leq$ x, и присваиваем каждому C[x] сумму значений предыдущего и текущего элементов.
\item Перебираем массив A с конца, помещаем каждое значение в новый отсортированный массив B по индексу, записанному в С. Делается это для текущего элемента A[x] путем присваивания A[x] в B[C[A[x]] и декрементирования C[A[x]] в том случае, если было несколько чисел равных по значению в исходном неотсортированном массиве.
\end{enumerate}

\vspace{\baselineskip}
\textbf{Пример сортировки подсчетом:}

\vspace{\baselineskip}
\includeimage{1}{images/167_1.pdf}

\vspace{\baselineskip}
\textbf{Дополнительная память}: O(n+k)

\vspace{\baselineskip}
\textbf{Временная сложность}: в худшем случае - O(n+k),в лучшем - O(n), в стандартном - O(n+k)

\vspace{\baselineskip}
\section*{Раздел 34.2: Реализация на псевдокоде} 

\vspace{\baselineskip}
Ограничения:

\vspace{\baselineskip}
\begin{enumerate}
\item Входные данные (массив для сортировки)
\item Количество элементов на входе (n)
\item Ключи в диапазоне 0..к-1 (к)
\item Счетчик (массив чисел)
\end{enumerate}

\vspace{\baselineskip}
Псевдокод:

\vspace{\baselineskip}
\begin{tcolorbox}
\begin{minted}{C} 
for x in input:
	count[key(x)] += 1
total = 0
for i in range(k):
	oldCount = count[i]
	count[i] = total
	total += oldCount
for x in input:
	output[count[key(x)]] = x
	count[key(x)] += 1
return output
\end{minted}
\end{tcolorbox}

\chapter*{Глава 35: Пирамидальная сортировка}
\section*{Раздел 35.1: Реализация на C\#} 

\vspace{\baselineskip}
\begin{tcolorbox}
\begin{minted}{C} 
public class HeapSort
{
	public static void Heapify(int[] input, int n, int i)
	{
		int largest = i;
		int l = i + 1;
		int r = i + 2;
		
		if (l < n && input[l] > input[largest])
			largest = l;
			
		if (r < n && input[r] > input[largest])
			largest = r;
			
		if (largest != i)
		{
			var temp = input[i];
			input[i] = input[largest];
			input[largest] = temp;
			Heapify(input, n, largest);
		}
	}
	public static void SortHeap(int[] input, int n)
	{
		for (var i = n - 1; i >= 0; i--)
		{
			Heapify(input, n, i);
		}
		for (int j = n - 1; j >= 0; j--)
		{
			var temp = input[0];
			input[0] = input[j];
			input[j] = temp;
			Heapify(input, j, 0);
		}
	}
	public static int[] Main(int[] input)
	{
		SortHeap(input, input.Length);
		return input;
	}
}
\end{minted}
\end{tcolorbox}

\section*{Раздел 35.2: Основная информация о пирамидальной
сортировке} 

\vspace{\baselineskip}
Пирамидальная сортировка — это метод сортировки сравнением,основанный на такой структуре данных как двоичная куча. Она похожа на сортировку выбором, где мы сначала ищем максимальный элемент и помещаем его в конец. Далее мы повторяем ту же операцию для оставшихся элементов.

\vspace{\baselineskip}
\textbf{Псевдокод для пирамидальной сортировки:}

\vspace{\baselineskip}
\begin{tcolorbox}
\begin{minted}{C} 
function heapsort(input, count)
	heapify(a,count)
	end <- count - 1
	while end -> 0 do
	swap(a[end],a[0])
	end<-end-1
	restore(a, 0, end)
	
function heapify(a, count)
	start <- parent(count - 1)
	while start >= 0 do
		restore(a, start, count - 1)
		start <- start - 1

\end{minted}
\end{tcolorbox}

\vspace{\baselineskip}
\textbf{Пример пирамидальной сортировки:}

\vspace{\baselineskip}
\includeimage{0.6}{images/170_1.pdf}

\vspace{\baselineskip}
\textbf{Дополнительная память:} O(1)

\textbf{Временная сложность:} O(nlog n)

\vspace{\baselineskip}

\chapter*{Глава 36: Сортировка циклом}
\section*{Раздел 36.1: Реализация на псевдокоде} 

\vspace{\baselineskip}
\begin{tcolorbox}
\begin{minted}{C} 
(input)
output = 0
for cycleStart from 0 to length(array) - 2
	item = array[cycleStart]
	pos = cycleStart
	for i from cycleStart + 1 to length(array) - 1
		if array[i] < item:
			pos += 1
	if pos == cycleStart:
		continue
	while item == array[pos]:
		pos += 1
	array[pos], item = item, array[pos]
	writes += 1
	while pos != cycleStart:
		pos = cycleStart
		for i from cycleStart + 1 to length(array) - 1
			if array[i] < item:
				pos += 1
		while item == array[pos]:
			pos += 1
		array[pos], item = item, array[pos]
		writes += 1
return outout

\end{minted}
\end{tcolorbox}

\chapter*{Глава 37: Четно-нечетная сортировка}
\section*{Раздел 37.1: Основная информация о четно-нечетной сортировке} 

\vspace{\baselineskip}
Четно-нечетная сортировка или сортировка по кирпичикам - это простой алгоритм сортировки, разработанный для использования на параллельных процессорах с локальным соединением. Работает он путем сравнения всех нечетных- четных проиндексированных пар соседних элементов в списке, и, если пара находится в неправильном порядке,то элементы переставляются. Следующий шаг повторяется для четных-нечетных проиндексированных пар. Затем нечетные-четные и четные-нечетные шаги чередуются до тех пор, пока список не будет отсортирован.

\vspace{\baselineskip}
\textbf{Псевдокод для четно-нечетной сортировки:}

\vspace{\baselineskip}
\begin{tcolorbox}
\begin{minted}{C} 
if n>2 then
	1. apply odd-even merge(n/2) recursively to the even subsequence a0, a2, ..., an-2 and to the
odd subsequence a1, a3, , ..., an-1
	2. comparison [i : i+1] for all i element {1, 3, 5, 7, ..., n-3}
else
	comparison [0 : 1]

\end{minted}
\end{tcolorbox}

\vspace{\baselineskip}
\textbf{На Википедии есть лучшая иллюстрация четно-нечетной сортировки:}

\vspace{\baselineskip}
\href{https://i.stack.imgur.com/FVktW.gif}{\underline{Четно-нечетная сортировка}}

\vspace{\baselineskip}
\textbf{Пример четно-нечетной сортировки:}

\newpage

\vspace{\baselineskip}
\includeimage{1}{images/173_1.pdf}

\vspace{\baselineskip}
\textbf{Реализация:}

\vspace{\baselineskip}
Используется C\# для реализации четно-нечетного алгоритма сортировки:

\vspace{\baselineskip}
\begin{tcolorbox}
\begin{minted}{Csharp} 
public class OddEvenSort
{
	private static void SortOddEven(int[] input, int n)
	{
		var sort = false;
		
		while (!sort)
		{
			sort = true;
			for (var i = 1; i < n - 1; i += 2)
			{
				if (input[i] <= input[i + 1]) continue;
				var temp = input[i];
				input[i] = input[i + 1];
				input[i + 1] = temp;
				sort = false;
			}
			for (var i = 0; i < n - 1; i += 2)
			{
				if (input[i] <= input[i + 1]) continue;
				var temp = input[i];
				input[i] = input[i + 1];
				input[i + 1] = temp;
				sort = false;
			}
		}
	}
	public static int[] Main(int[] input)
	{
		SortOddEven(input, input.Length);
		return input;
	}
}
\end{minted}
\end{tcolorbox}


\vspace{\baselineskip}
\textbf{Дополнительная память:} O(n) 

\textbf{Временная сложность:} O(n) 

\chapter*{Глава 38: Сортировка выбором}
\section*{Раздел 38.1: Реализация на Elixir}

\vspace{\baselineskip}
\begin{tcolorbox}
\begin{minted}{C} 
defmodule Selection do

	def sort(list) when is_list(list) do
		do_selection(list, [])
	end
	
	def do_selection([head|[]], acc) do
		acc ++ [head]
	end
	
	def do_selection(list, acc) do
		min = min(list)
		do_selection(:lists.delete(min, list), acc ++ [min])
	end
	
	defp min([first|[second|[]]]) do
		smaller(first, second)
	end
	
	defp min([first|[second|tail]]) do
		min([smaller(first, second)|tail])
	end
	
	defp smaller(e1, e2) do
		if e1 <= e2 do
		  e1
		else
		  e2
		end
	end
end
Selection.sort([100,4,10,6,9,3])
|> IO.inspect

\end{minted}
\end{tcolorbox}

\section*{Раздел 38.2: Основная информация о сортировке выбором}

\vspace{\baselineskip}
Сортировка выбором - это алгоритм сортировки, в частности сортировка сравнения на месте.Сортировка имеет временную сложность O(n2), что делает ее неэффективной в больших списках и обычно работает хуже, чем сортировка вставкой. Сортировка выбором отличается простотой и имеет преимущества в производительности по сравнению с более сложными алгоритмами в определенных ситуациях, в частности там , где дополнительная память ограничена.

\vspace{\baselineskip}
Алгоритм разделяет входной список на две части: на подсписок уже отсортированных элементов, который строится в начале списка слева направо, и на подсписок оставшихся элементов, которые занимают оставшуюся часть списка.Изначально отсортированный подсписок пуст, а несортированный подсписок - это весь входной список. Алгоритм работает следующим образом: находится наименьший( или наибольший, в зависимости от порядка сортировки) элемент в неотсортированном подсписке, обменивается с крайним левым элементом из неотсортированного подсписка, затем граница отсортированного подсписка перемещается направо.

\vspace{\baselineskip}
\textbf{Псевдокод для сортировки выбором:}

\vspace{\baselineskip}
\begin{tcolorbox}
\begin{minted}{C} 
function select(list[1..n], k)
 for i from 1 to k
	minIndex = i
	minValue = list[i]
	for j from i+1 to n
		if list[j] < minValue
			minIndex = j
			minValue = list[j]
	swap list[i] and list[minIndex]
return list[k]

\end{minted}
\end{tcolorbox}

\vspace{\baselineskip}
Визуализация сортировки выбором:

\vspace{\baselineskip}
\href{https://i.stack.imgur.com/LZepY.gif}{\underline{Сортировка выбором}}

\vspace{\baselineskip}
\textbf{Пример Сортировки Выбором:}

\newpage

\vspace{\baselineskip}
\includeimage{1}{images/177_1.pdf}

\vspace{\baselineskip}
\textbf{Дополнительная память:} O(n)

\textbf{Временная сложность:} O(n\^2)

\section*{Раздел 38.3: Реализация Сортировки выбором на C\#}

\textbf{Используется C\# для реализации алгоритма сортировки выбора:}

\vspace{\baselineskip}
\begin{tcolorbox}
\begin{minted}{Csharp} 
public class SelectionSort
{
	private static void SortSelection(int[] input, int n)
	{
		for (int i = 0; i < n - 1; i++)
		{
			var minId = i;
			int j;
			for (j = i + 1; j < n; j++)
			{
				if (input[j] < input[minId]) minId = j;
			}
			var temp = input[minId];
			input[minId] = input[i];
			input[i] = temp;
		}
	}
	public static int[] Main(int[] input)
	{
		SortSelection(input, input.Length);
		return input;
	}
}
\end{minted}
\end{tcolorbox}